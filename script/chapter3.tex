\chapter{Supersymmetry}
\section{The hierarchy problem and its SUSY solution}

Problem is that scalar (Higgs) section of the SM is not stable against radiative corrections!

\paragraph{Two point function of electron} (related to electron mass): a well-behaved example
\begin{align}
   &\feynmandiagram[horizontal=a to v1, layered layout]{
      a[particle=\(e\)] --[fermion, momentum'=\(q\)] v1 --[fermion, momentum'=\(k\)] v2 --[fermion, momentum'=\(q\)] b[particle=\(e\)],
      v1 --[half left, photon, momentum={\(q-k\)}] v2,
   }; \notag \\
   \pi^{ee}(q\rightarrow 0) &=  \int \frac{\dd[4]{k}}{(2\pi)^4} (ie\gamma_\mu) \frac{-i g^{\mu\nu}}{k^2} \frac{i}{\slashed{k}-m_e} (ie \gamma_\nu) \notag \\
                            &= - e^2 \int \frac{\dd[4]{k}}{(2\pi)^4} \frac{1}{k^2 (k^2 - m_e^2)} \gamma_\mu (\slashed{k} + m_e )\gamma^\mu \notag \\
                            &= -e^2 m_e \int \frac{\dd[4]{l}}{(2\pi)^4} \frac{\gamma_\mu \gamma^\mu}{k^2 (k^2 - m_e^2)} \label{3.2}
\end{align}

Note
\begin{itemize}
   \item Correction vanished if tree-level $m_e \rightarrow 0$
   \item Correction "only" diverge logarithmically
      \begin{equation}
         \delta m_e \sim m_e \frac{\alpha}{\pi} \ln \frac{\Lambda}{m_e}  \label{3.3}
      \end{equation}
      with $\Lambda$ a cut-off parameter, e.g.~take $\Lambda = M_{pl} = \SI{2.4e18}{\giga\eV} = \num{4.7e21} \cdot m_e$
      \begin{equation}
         \delta m_e \cong \frac{m_e}{8}
      \end{equation}
\end{itemize}

Small electron mass remains small! Reason for this is that setting $m_e \rightarrow 0$ increases the symmetry of the theory. Lagrangian becomes invariant under global chiral $\Uni (1)$ transformation
\begin{equation}
   \psi \rightarrow e^{i \chi \gamma_5 } \psi = (\cos \chi + i \gamma_5 \sin \chi) \psi
\end{equation}
with $\chi = \text{const.}$. The bilinear $\overline{\psi} \gamma_\mu \psi$ is invariant under this transformation
\begin{align}
   \overline{\psi} \gamma_\mu \psi &\rightarrow \overline{\psi} (\cos \chi - i \overline{\gamma}_5 \sin \chi) \gamma_\mu (\cos \chi + i \gamma_5 \sin \chi) \psi \notag \\
                                   &= \overline{\psi} (\cos \chi + i \gamma_5 \sin \chi) (\cos \chi - i \gamma_5 \sin \chi) \gamma_\mu \psi \notag \\
                                   &= \overline{\psi} ((\cos \chi)^2 - (i \gamma_5 \sin \chi)^2) \gamma_\mu \psi = \overline{\psi} \gamma_\mu \psi \label{3.6}
\end{align}
But $\overline{\psi} \psi$ not
\begin{equation}
   \overline{\psi} \psi \rightarrow \overline{\psi} (\cos \chi  i \gamma_5 \sin \chi)^2 \psi \neq \overline{\psi} \psi \label{3.7}
\end{equation}
Fermion masses or Yukawa couplings break chiral symmetry. Conversely, $f^{(e)}\rightarrow 0$ restores chiral symmetry.

Small fermion mass is technically natural (t' Hooft), as it will remain small in all orders in perturbation theory!

\paragraph{Two-point function of Higgs boson}
\begin{align}
   &\feynmandiagram[horizontal=i to v1, layered layout]{
      i[particle=\(h\)] --[scalar] v1 --[half left, fermion, edge label=\(t\)] v2 --[scalar] f[particle=\(h\)],
      v2 --[half left, fermion, edge label=\(t\)] v1,
   }; \notag \\
   \Pi^h_t (0) &= - 3 \int \frac{\dd[4]{k}}{(2\pi)^4} \tr[ - i \frac{f^{(t)}}{\sqrt{2}} \frac{i}{\slashed{k} - m_t} \left( -i \frac{f^{(t)}}{\sqrt{2}} \right) \frac{i}{\slashed{k} - m_t} ] \notag \\
               &= -\frac{3}{2} f^{(t) \, 2} \int \frac{\dd[4]{k}}{(2\pi)^4} \frac{1}{(k^2 - m_t^2)^2 } \tr[ (\slashed{k} + m_t) (\slashed{k} + m_t)] \notag \\
               &= -6 f^{(t) \, 2} \int \frac{\dd[4]{k}}{(2\pi)^4} \frac{k^2 + m_t^2}{(k^2 - m_t^2)^2} \label{3.5}
\end{align}
where negative sign comes from closed fermion loop and $3$ from colors.

Note
\begin{itemize}
   \item Result diverges quadratically!
   \item Correction does not depend on tree-level Higgs mass
      \begin{equation}
         \delta m_h^2 \sim \frac{3 f{(t)\, 2}}{8\pi^2 } \Lambda^2 \label{3.6}
      \end{equation}
\end{itemize}
For $\Lambda=M_{pl}$: 
\begin{equation}
   \delta m_h^2 \sim \num{1e30} m^2_{h, \text{phys}}  \label{3.7}
\end{equation}
Since 
\begin{equation}
   m^2_{h, \text{phys}} = m^2_{h,0} + \delta m^2_h \label{3.8}
\end{equation}
it needs cancellation to $1$ part in $10^{30}$. This extreme \textit{finetuning} is not \textit{natural}. Reason is that $m_h \rightarrow 0$ does not increase symmetry of SM (at quantum level). Following t' Hooft, Susskind, Weinberg: $m_h$ should be close to highest scale when SM is applicable! If there exists physical scale $M \gg m_h$, we expect finite corrections $\sim \alpha/\pi M^2$!

\paragraph{Example} for a successful cure of a finetuning problem

In standard cosmology, the energy density of the Universe in units of the critical energy density $\Omega = \rho / \rho_\text{crit}$ is time-dependent.
\begin{equation*}
   \Omega(t) - 1 \sim \left[ \Omega(t_0) - 1 \right] \cdot \left[ \frac{T(t_0)}{T(t)} \right]^{\beta}
\end{equation*}
with $\beta=1$ or $2$.

Now $\Omega(t_0) = 1.00 \pm 0.03$. If $ T \gg T_0 (\sim \SI{1e-4}{\eV})$
\begin{equation*}
   |\Omega(t_0) -1 | \gg |\Omega(t) - 1|
\end{equation*}
Need $|\Omega(t) - 1| \gg 1$ at $T \gg T_0$: fine-tuning of initial conditions!

This fine-tuning is solved by postulating very early epoch of exponential growth called "inflation"! Generically predicts $|\Omega(t_0) - 1| \ll 1$. Predictions for fluctuations in CMB have been confirmed! Nature really abhors finetuning (?).

Counter-example is the potential equivalence of vacuum energy and cosmological constant. Naively it corresponds to zero-point function, which has a quartic divergence in field theory (mode sum, $\sim \int \frac{\dd[4]{k}}{(2\pi)^4}$). It gets reduced to quadratic divergence in softly broken SUSY. There is no (convincing) solution known. We may need theory of quantum gravity. 

Back to (\ref{3.6}), (\ref{3.8}): Absence of serious finetuning, i.e.~
\begin{equation}
   \delta e_h^2 \lessapprox m_{h, \text{phys}}^2
   \label{3.9}
\end{equation}
implies
\begin{equation}
   \Lambda \lessapprox \order{1} \si{\tera \eV}
   \label{3.10}
\end{equation}
SM must be replaced by a different theory at (few) $\si{\tera\eV}$ scale!

One possibility is that SM to theory without elementary scalars, i.e.~$h$ is composite. This is the idea behind technicolor theories. It has following problems
\begin{itemize}
   \item "New physics" is not sufficiently decoupling. It generically expects sizeable effects in precision experiments and they have not been seen.
   \item Higgs mechanism also responsible for fermion masses in SM. It becomes difficult to generate a large top mass without generating much too large FCNC!
   \item New interactions need to be very strongly coupled, thus difficult to perform reliable calculations.
\end{itemize}
People keep trying but there is no convincing solution known.

Here we consider second option, supersymmetry: quadratic divergences cancel order by order. To this end, we need superpartners for each SM particle, with "same" interactions, but spin differing by $1/2$ unit.

Here to consider "stops" $\SU(2)_L$ doublet $\tilde t_L$, and singlet $\tilde t_R$. They both are complex scalars and color triplets.
\begin{equation}
   \lag_{\tilde t h} = \tilde \lambda_t |\phi^0|^2 \left( |\tilde t_L|^2 + |\tilde t_R|^2 \right) + \left[ f^{(t)} A_t \phi^0 \tilde t_L \tilde t_R^* + h.c. \right]
   \label{3.11}
\end{equation}
It gives new diagrams
\begin{equation*}
   \begin{tikzpicture}[scale=1, transform shape, baseline=(i.base)]
      \begin{feynman}
         \vertex (i) {\(h\)};
         \vertex[right=1.5cm of i, label=270:{\(i\tilde \lambda_t\)}, dot] (v) ;
         \vertex[above=1.5cm of v, label=90:{\(\tilde t_{L,R}\)}] (x) ;
         \vertex[right=1.5cm of v] (f) {\(h\)};
         \diagram*{
            (i) --[scalar] (v) --[scalar] (f),
            (v) --[scalar, half left] (x) --[scalar, half left] (v),
         };
      \end{feynman}
   \end{tikzpicture}
   +    
   \begin{tikzpicture}[scale=1, transform shape, baseline=(i.base)]
      \begin{feynman}
         \vertex (i) {\(h\)};
         \vertex[right=1.5cm of i, label=235:{\(i\sqrt{2}\tilde \lambda_t v\)}] (v1) ;
         \vertex[right=1.5cm of v1] (v2) ;
         \vertex[right=1.5cm of v2] (f) {\(h\)};
         \diagram*{
            (i) --[scalar] (v1)[dot] --[scalar, half left, edge label={\(\tilde t_{L,R}\)}] (v2) --[scalar] (f),
            (v1) --[scalar, half right, edge label'={\(\tilde t_{L,R}\)}] (v2),
         };
      \end{feynman}
   \end{tikzpicture}
   +
   \begin{tikzpicture}[scale=1, transform shape, baseline=(i.base)]
      \begin{feynman}
         \vertex (i) {\(h\)};
         \vertex[right=1.5cm of i, label=235:{\(f^{(t)}A_t/\sqrt{2}\)}] (v1) ;
         \vertex[right=1.5cm of v1] (v2) ;
         \vertex[right=1.5cm of v2] (f) {\(h\)};
         \diagram*{
            (i) --[scalar] (v1)[dot] --[scalar, half left, edge label={\(\tilde t_{L,R}\)}] (v2) --[scalar] (f),
            (v1) --[scalar, half right, edge label'={\(\tilde t_{R,L}\)}] (v2),
         };
      \end{feynman}
   \end{tikzpicture}
\end{equation*}
\begin{align}
   \Pi^h_{\tilde t} (0) &= 3 \int \frac{\dd[4]{k}}{(2\pi)^4} \Bigg[ i \tilde \lambda_t \left( \frac{i}{k^2 - m^2_{\tilde{t}_L}} + \frac{i}{k^2 - m^2_{\tilde t_R}} \right) 
                         + (i\sqrt{2} \tilde \lambda_t v)^2 \left( \frac{i^2}{(k^2 - m^2_{\tilde t_L})^2} + \frac{i^2}{(k^2 - m^2_{\tilde t_R})^2} \right) \notag \\
                        &\quad + 2 \left( \frac{if^{(t)} A_t}{\sqrt{2}} \right)^2 \frac{i^2}{(k^2 - m^2_{\tilde t_L})(k^2 - m^2_{t_R})} \Bigg] \notag \\
                        &= 3 \int \frac{\dd[4]{k}}{(2\pi)^4} \Bigg[ - \tilde \lambda_{t} \left( \frac{1}{k^2 - m^2_{\tilde t_L}} + \frac{1}{k^2 - m^2_{\tilde t_R}} \right) 
                         + 2 (\tilde{\lambda}_t v)^2 \left( \frac{1}{(k^2 - m^2_{\tilde{t}_L})^2} + \frac{1}{(k^2 - m^2_{\tilde{t}_R})^2} \right) \notag \\
                        &\quad + \left(f^{(t)} A_t\right)^2 \frac{1}{(k^2 - m^2_{\tilde t_L})(k^2 - m^2_{\tilde t_R})} \Bigg] \label{3.12}
\end{align}
Only first term in (\ref{3.12}) has quadratic divergence.

The quadratic divergence gets cancelled from (\ref{3.5}) if
\begin{equation}
   \tilde \lambda_t = - f^{(t)2}
   \label{3.13}
\end{equation}
Note that $\tilde \lambda_t < 0$ for potential to be bounded from below!
\begin{align}
   \Pi_{t + \tilde{t}}^h(0) &= 3 f^{(t)2} \cdot \int \frac{\dd[4]{k}}{(2\pi)^4} \Bigg[ -2 \frac{k^2 + m_t^2}{(k^2 - m^2_t)^2} + \frac{1}{k^2 - m_{\tilde t_L}^2} + \frac{1}{k^2 - m^2_{\tilde t_R}} + 2 (f^{(t)}v)^2 \left( \frac{1}{(k^2 - m_{\tilde t_L})^2} + \frac{1}{(k^2 - m^2_{\tilde t_R})^2} \right) \notag \\
                            &\quad+ A^2_t \frac{1}{(k^2 - m_{\tilde t_L}^2)(k^2 - m^2_{\tilde t_R})} \Bigg] \notag \\
                            &=3 f^{(t)2} \int \frac{\dd[4]{k}}{(2\pi)^4} \Bigg[ \frac{m^2_{\tilde t_L} - m^2_t}{(k^2 - m_t^2)(k^2 - m^2_{\tilde t_L})} + \frac{m^2_{\tilde t_R}-m^2_t}{(k^2 - m_t^2)(k^2 - m^2_{\tilde t_R})} + 2m^2_t \left( \frac{1}{(k^2 - m^2_{\tilde t_L})^2} + \frac{1}{(k^2 - m^2_{\tilde t_R})^2} - \frac{2}{(k^2 - m^2_t)^2} \right)\notag \\
                            &\quad + A^2_t \frac{1}{(k^2 - m^2_{\tilde t_L})(k^2 - m^2_{\tilde t_R})} \Bigg] \label{3.14} 
\end{align}

Note that 
\begin{itemize}
   \item equation (\ref{3.14}) is only logarithmically divergent!
   \item if 
      \begin{align}
         m_t &= m_{\tilde t_L} = m_{\tilde t_R} \label{3.15a}\\
         A_t &= 0 \label{3.15b}
      \end{align}
      then
      \begin{equation}
         \pi^h_{t + \tilde{t}}(0) =0 \label{3.16}
      \end{equation}
\end{itemize}
More generally
\begin{equation}
   \delta m_h^2 \sim \frac{3 f^{(t)2}}{8\pi^2} \cdot \ln \frac{\Lambda}{m_n} \cdot [(m^2_{\tilde t} - m^2_t), A^2_t] \label{3.17}
\end{equation}
So we want $m_{\tilde{t}}, |A_t| \lessapprox \order{1}\si{\tera \eV}$!

In order to enforce cancellation of quadratic divergence in all orders of perturbation theory, and for all interactions, we need to supersymmetrize the entire SM, i.e.~we need a superpartner for each SM particle: doubling of particle spectrum!

Recall that chiral symmetry also double fermion spectrum! It is also helpful to remove divergences.

In classical EM, non-relativistic QM: $\delta m_e \sim e^2 \cdot \Lambda$. In QFT, this becomes a logarithmic divergence, but needs positrons!

\section{Grassmann variables}
SUSY connects (commuting) bosons with (anti-commuting) fermions. Most elegant "superfield" formalism is based on "supersymmetrization of space-time". To that end, introduce anti-commuting "Grassmann" coordinates.

For one complex $\theta$, with conjugate $\bar{\theta}$ ($\bar{\bar{\theta}}=\theta$), we postulate
\begin{subequations}
   \label{3.18}
\begin{align}
   \acomm{\theta}{\theta} &= \acomm{\bar\theta}{\bar\theta} = 0 \label{3.18a}\\
   \acomm{\theta}{\bar\theta} &= 0 \label{3.18b}
\end{align}
\end{subequations}
These two relations generate the Grassmann algebra. From them,
\begin{subequations}
\label{3.19}
\begin{align}
   \theta^2 &= \bar{\theta}^2 = 0 \label{3.19a} \\
   \theta \bar\theta &= - \bar \theta \theta \label{3.19b}
\end{align}
\end{subequations}
With (\ref{3.19}), we can exactly expand an analytic function
\begin{equation}
   f(\theta) = f_0 + f_1 \theta \label{3.20}
\end{equation}
with $f_0, f_1 \in \Co$.

Define derivative
\begin{equation}
   \dv{\theta} f(\theta) = f_1 \label{3.21}
\end{equation}
and $\theta f(\theta) = \theta f_0$. 

Equation (\ref{3.20}), (\ref{3.21}) also hold with $\theta \rightarrow \bar \theta$ with $f_0 \rightarrow \bar{f}_0, f_1 \rightarrow \bar{f}_1$. General function
\begin{equation}
   f(\theta, \bar\theta) = f_0 + f_1 \theta + f_2 \bar\theta + f_3 \theta \bar\theta \label{3.22}
\end{equation}
with $\pdv{\theta} f = f_1 + f_3 \bar\theta$, etc..

Integration rules
\begin{subequations}
   \label{3.23}   
\begin{align}
   \int \dd{\theta} \theta &= \int \dd{\bar\theta} \bar{\theta} = 1 \label{3.23a} \\
   \int \dd{\theta} &= \int \dd{\bar\theta} = \int \dd{\theta} \pdv{\theta} f(\theta,\bar\theta) = \int \dd{\bar\theta} \pdv{\theta} f(\theta,\bar\theta) = 0 \label{3.23c} \\
   \int \dd{\theta} \dd{\bar\theta} f(\theta, \bar\theta) &= f_3 \label{3.23b}
\end{align}
\end{subequations}

Integral over Grassmann number is linear
\begin{equation}
   \int \dd{\theta} \left[ \alpha f(\theta) + \beta g(\theta) \right] = \alpha \int \dd{\theta} f(\theta) + \beta \int \dd{\theta} g(\theta) \label{3.24}
\end{equation}

It has translational invariance
\begin{equation*}
   \int \dd{\theta_i} f(\theta_i + \theta_k) = \int \dd{\theta_i} f(\theta_i)
\end{equation*}
independent of $\theta_k$, if $\int \dd{\theta_i} \theta_k = \delta_{ik}$.

Grassmann $\delta$-functions
\begin{subequations}
\label{3.26}   
\begin{align}
   \int \dd{\theta} \delta(\theta) f(\theta) \stackrel{(\ref{3.20})}{=} f_0  \label{3.26a}\\
   \int \dd{\theta} \delta(\theta - \theta') f(\theta) = f(\theta')  \label{3.26b}
\end{align}
\end{subequations}
$\delta$-function has the explicit representation
\begin{equation}
   \delta(\theta - \theta') = \theta - \theta' \label{3.27}
\end{equation}
i.e.~$\delta(\theta) = \theta$.
(\ref{3.26a}) follows from (\ref{3.23a}) and (\ref{3.23c})
\begin{align*}
   \int \dd{\theta} (\theta - \theta') (f_0 + f_1 \theta) &= \int \dd{\theta} \theta f_0 + \int \dd{\theta} \theta f_1 \theta - \int \dd{\theta} \theta' f_0 - \int \dd{\theta} \theta' f_1 \theta  \\
                                                          &= f_0 + 0 + 0 + f_1\theta' \\
                                                          &= f(\theta')
\end{align*}
\section{Algebraic aspects: SUSY algebra and supermultiplets}
Supersymmetry must be spacetime symmetry (like Lorentz symmetry), not internal symmetry (like Yang-Mills gauge symmetry). To see this, consider $2\pi$ rotation operator $U_{2\pi}$ with
\begin{subequations}
   \label{3.28}
\begin{align}
   U_{2\pi} \ket{\text{boson}} &= \ket{\text{boson}} \label{3.28a} \\
   U_{2\pi} \ket{\text{fermion}} &= -\ket{\text{fermion}} \label{3.28b}
\end{align}
\end{subequations}
Supercharge $Q$ transforms bosons into fermions and vice versa
\begin{subequations}
   \label{3.29}
\begin{align}
   Q \ket{\text{boson}} &= \ket{\text{fermion}} \label{3.29a} \\
   Q \ket{\text{fermion}} &= \ket{\text{boson}} \label{3.29b}
\end{align}
\end{subequations}


Then
\begin{align*}
   U_{2\pi} Q U^{-1}_{2\pi} \ket{\text{fermion}} & \stackrel{(\ref{3.28b})}{=} - U_{2\pi} Q U^{-1}_{2\pi} U_{2\pi} \ket{\text{fermion}} \\
                                                 &= - U_{2\pi} Q \ket{\text{fermion}} \\
                                                 &\stackrel{(\ref{3.29b})}{=} - U_{2\pi} \ket{\text{boson}} \\
                                                 & \stackrel{(\ref{3.28a})}{=} - \ket{\text{boson}} \stackrel{(\ref{3.29b})}{=} - Q \ket{\text{fermion}}
\end{align*}
Thus $U_{2\pi} Q U^{-1}_{2\pi} = - Q$, it ($Q$) behaves like spinorial operator. We will need to expand Poincaré algebra to include anti-commutator!

Inhomogeneous Lorentz transformation (Poincaré)
\begin{equation}
   x^\mu \rightarrow x'^{\mu}  = (\delta^\mu_\nu + \omega^\mu_\nu) x^\nu + a^\mu \label{3.30}
\end{equation}
with $\omega_{\mu\nu} = - \omega_{\nu\mu}$. The corresponding unitary operator $U(a) = e^{i a\cdot \underline{P}}$
\begin{equation}
   U(\Lambda) = \exp(- \frac{i}{2} \omega_{\mu\nu} M^{\mu\nu}) \label{3.31}
\end{equation}

Poincaré algebra
\begin{subequations}
   \label{3.32}
\begin{align}
   \comm{\underline{P}_\mu}{\underline{P}_\nu} &= 0 \label{3.32a} \\
   \comm{M_{\mu\nu}}{\underline{P}_\rho} &= i (g_{\nu\rho} \underline{P}_\mu - g_{\mu\rho} P_\nu) \label{3.32b} \\
   \comm{M_{\mu\nu}}{M_{\rho\sigma}} &= -i (g_{\mu\sigma}M_{\nu\rho} - g_{\nu\sigma} M_{\mu\rho} - g_{\nu\rho} M_{\mu\sigma} + g_{\mu\rho} M_{\nu\sigma}) \label{3.32c}
\end{align}
\end{subequations}

Explicit spinorial realization
\begin{equation}
   \Sigma_{\mu\nu} = \frac{i}{4} \comm{\gamma_\mu}{\gamma_\nu}
\end{equation}
thus
\begin{equation}
   M_{\mu\nu} = - x_\mu \underline{P}_\nu + x_\nu \underline{P}_\mu + \Sigma_{\mu\nu} \label{3.35}
\end{equation}

Physically $P_\mu$ is the $4$-momentum and $M_{\mu\nu}$ the total angular momentum, and $\Sigma_{\mu\nu}$ is the spin contribution.

In chiral (Weyl) representation (\ref{0.5})
\begin{equation}
   \Sigma_{\mu\nu} = \begin{pmatrix} \sigma^{\mu\nu} & 0 \\ 0 & \bar{\sigma}_{\mu\nu}\end{pmatrix} \label{3.36a}
\end{equation}
with
\begin{align}
   \sigma^{\mu\nu} &= \frac{i}{4} \left( \sigma^\mu \bar\sigma^\nu - \sigma^\nu \bar \sigma^\mu \right) \label{3.36b} \\
   \bar\sigma^{\mu\nu} &= \frac{i}{4} \left( \bar\sigma^\mu \sigma^\nu - \bar\sigma^\nu \sigma^\mu \right) \label{3.36c}
\end{align}
where
\begin{equation}
   \sigma^\mu = (\id_{2}, \pmb{\sigma}); \quad \bar\sigma^{\mu} = (\id_{2}, - \pmb{\sigma}) \label{3.37}
\end{equation}
and
\begin{equation}
   \bar\sigma^{\mu\nu} = \sigma^{\mu\nu\dagger}
\end{equation}

They have the following properties
\begin{subequations}
\begin{align}
   \acomm{\sigma^\mu}{\bar\sigma^\nu} &= \acomm{\bar\sigma^\mu}{\sigma^\nu} = 2g^{\mu\nu} \id_{2\times 2} \label{3.39} \\
   \tr(\sigma^\mu \bar\sigma^\nu) &= 2 g^{\mu\nu} \label{3.40} \\
   \tr(\sigma^{\mu\nu}\sigma^{\alpha\beta}) &= \frac{1}{2} \left( g^{\mu\alpha} g^{\nu\beta} - g^{\mu\beta} g^{\nu\alpha} \right) + \frac{i}{2} \epsilon^{\mu\nu\alpha\beta} \label{3.41}
\end{align}
\end{subequations}
$\epsilon$ is the rank-$4$ totally anti-symmetric tensor
\begin{subequations}
\begin{align}
   \epsilon^{0123} &= - \epsilon_{0123} = -1 \label{3.42} \\
   \epsilon^{\mu\nu\alpha\beta} \sigma_{\alpha\beta} &= 2i \sigma^{\mu\nu}; \quad \epsilon^{\mu\nu\alpha\beta} \bar\sigma_{\alpha\beta} = -2i \bar\sigma^{\mu\nu} \label{3.43}
\end{align}
\end{subequations}

\paragraph{Coleman-Mandula theorem}
Note that $M_{\mu\nu}$, and $P_\mu$ are bosonic generators. Restricting to such generators is the so-called \textit{Coleman-Mandula theorem}.
Consider the full Lie-algebra of symmetries of the $S$-matrix. In addition to $P_\mu$, $M_{\mu\nu}$, this contains bosonic generators $t^a$ with
\begin{equation}
   \comm{t^a}{t^b} = i f^{abc} t^c \label{3.44}
\end{equation}
where $f^{abc}$ is the structure constants.

Requiring
\begin{itemize}
   \item A unique ground state
   \item Massive particles in finite-dimensional representation of Lorentz group
\end{itemize}
Thus
\begin{equation}
   \comm{t^a}{P_\mu} = \comm{t^a}{M_{\mu\nu}} = 0
\end{equation}
for all $a, \mu, \nu$, i.e.~$t^a$ describe \textit{purely internal symmetry} (e.g.~Yang-Mills gauge symmetries). In other word, most general \textit{bosonic} symmetry is direct product of Lorentz symmetry and (possibly quite complicated) YM gauge symmetry.

However, $Q$ is \textit{fermionic}. Thus we need ($Z_2$-)graded algebra, with \textit{even} and \textit{odd} elements, and defined through commutators and anti-commutators
\begin{subequations}
\begin{align}
   \comm{\text{even}}{\text{even}} &= \text{even} \label{3.46a} \\
   \comm{\text{even}}{\text{odd}} &= \text{odd} \label{3.46b} \\
   \acomm{\text{odd}}{\text{odd}} &= \text{even} \label{3.46c}
\end{align}
\end{subequations}
where the odd generators belong to the representation $(\frac{1}{2}, 0)$ and $(0, \frac{1}{2})$ of the homogeneous Lorentz group and the even generators are a direct sum of the Poincaré and other symmetry generators (i.e.~the latter two sets of generators mutually commute).
Equations (\ref{3.32a}), (\ref{3.32b}), and (\ref{3.32c}) are examples for structure of type (\ref{3.46a}). Supercharges are \textit{odd} generators.

To discuss how to fit $Q$ into Lorentz group, introduce
\begin{align}
   J_p &= \frac{1}{2} \epsilon_{pvs} M_{vs}; \quad K_p = - M_{0p} \label{3.47} \\
   J_p^\pm &= \frac{i}{2} (J_p \pm i K_p) \label{3.48}
\end{align}

Allows to re-write (\ref{3.32c}), which defines the algebra of homogeneous Lorentz group $\SO(1,3)$, as homomorphic to $\SU(2)_+ \times \SU(2)_-$ algebra
\begin{align*}
   \comm{J^{\pm}_p}{J^\pm_q} &= i \epsilon_{pqr} J^\pm_r \\
   \comm{J^+_p}{J^-_q} &= 0
\end{align*}
Finite-dimensional representation of homogeneous Lorentz group can equivalently be written as $(j_1, j_2)$. $j_{1,2}$ are (half) integer "spin quantum numbers", eigenvalues of $J_3^\pm$. $j_1$ refers to $\SU(2)_+$ and $j_2$ to $\SU(2)_-$.
\paragraph{Examples}
\begin{itemize}
   \item $(\frac{1}{2}, 0)$: left-chiral spin-$\frac{1}{2}$ fermion
   \item $(0, \frac{1}{2})$: left-chiral spin-$\frac{1}{2}$ fermion
   \item $(\frac{1}{2}, \frac{1}{2})$: spin-$1$ vector ($\neq (\frac{1}{2}, 0) + (0, \frac{1}{2})$)
   \item $(0,0)$: spin-$0$ scalar
\end{itemize}
Saw above (\ref{3.29a}) and (\ref{3.29b}) ff: supercharges behave like spinors under $2\pi$ rotation. Simplest consistent ansatz in $d=4$ is to introduce one Majorana spinor!
\begin{align}
   Q &= Q^c; \quad Q_a = C_{ab} \bar{Q}_b \quad \text{see \ref{1.42}}\label{3.50} \\
   \comm{M_{\mu\nu}}{Q_a} &= - \left( \Sigma_{\mu\nu} \right)_{ab} Q_b \label{3.51}
\end{align}
with $a,b \in \{1,2,3,4\}$ $4$-spinor (Dirac) indices. It is example of structure (\ref{3.46b}). Second example
\begin{equation}
   \comm{Q_a}{P_\mu} = (c_1 \gamma_\mu + c_2 \gamma_\mu \gamma_5)_{ab} Q_b \label{3.52}
\end{equation}
with $c_1, c_2 \in \Co$. It must be true, since $Q$ is the only odd generator. RHS must be odd, carry one free Lorentz index.

\begin{equation}
   \Rightarrow \comm{\bar{Q}}{P_\mu} = c_1^* \bar Q \gamma_\mu + c_2^* \bar Q \gamma_\mu \gamma_5 \label{3.53}
\end{equation}
with $P_\mu$ hermitian, $\bar\gamma_\mu = \gamma_0 \gamma_\mu$ and $\overline{\gamma_\mu \gamma_5} =\gamma_0 \gamma_\mu \gamma_5$.
\begin{equation*}
   \stackrel{(\ref{3.50})}{\Rightarrow} Q = C \bar Q^T \rightarrow \bar Q = - Q^T C^{-1}
\end{equation*}
Plug it into (\ref{3.53})
\begin{align}
   - \comm{Q^T}{P_\mu} C^{-1} &= -c_1^* Q^T C^{-1} \gamma_\mu C C^{-1} - c_2^* Q^T C^{-1} \gamma_\mu \gamma_5 C C^{-1} \notag \\
                              & \stackrel{(\ref{1.42})}{=} + c_1^* Q^T \gamma_\mu^T C^{-1} - c_2^* Q^{T} (\gamma_\mu \gamma_5)^T C^{-1} \notag \\
   \Rightarrow \comm{Q^T}{P_\mu} &= -c_1^* Q^T \gamma_\mu^T + c_2^* Q^T (\gamma_\mu \gamma_5)^T \label{3.54}
\end{align}
(\ref{3.54}) must be transposed of (\ref{3.52})
\begin{align}
   \begin{split}
      c_1^* &= - c_1 \\
      c_2^* &= c_2
   \end{split}\label{3.55}
\end{align}
(\ref{3.52}) must be consistent with Lorentz algebra. They must satisfy Jacobi identity 
\begin{align*}
   \comm{P_\mu}{\comm{P_\nu}{Q}} + \comm{P_\nu}{\comm{Q}{P_\mu}} + \comm{Q}{\cancel{\comm{P_\mu}{P_\nu}}} = 0 \\
   \stackrel{(\ref{3.52})}{\Rightarrow}\comm{P_\mu}{-(c_1 \gamma_\nu + c_2 \gamma_\nu \gamma_5)Q} + \comm{P_\nu}{(c_1 \gamma_\mu + c_2 \gamma_\mu \gamma_5 )Q} = 0 \\
   \stackrel{(\ref{3.52})}{\Rightarrow} (c_1 \gamma_\nu + c_2 \gamma_\nu \gamma_5)(c_1 \gamma_\mu + c_2 \gamma_\mu \gamma_5) Q  - (c_1 \gamma_\nu + c_2 \gamma_\nu \gamma_5)(c_1 \gamma_\nu + c_2\gamma_\nu \gamma_5) Q = 0 \\
   \Rightarrow \left[ c_1^2 \gamma_\nu \gamma_\mu + c_1 c_2 (\gamma_\nu \gamma_\mu \gamma_5 + \gamma_\nu \gamma_5 \gamma_\mu) + c_2^2 \gamma_\nu \gamma_5 \gamma_\mu \gamma_5 - (\mu\leftrightarrow \nu) \right] Q = 0
\end{align*}
for all $\mu,\nu$! Thus
\begin{equation}
   c_1^2 = c_2^2 \label{3.56}
\end{equation}
From (\ref{3.55}), $c_1^2 \leq 0$ and $c_2^2 \geq 0$, $\Rightarrow c_1 = c_2 = 0$.
\begin{equation}
   \comm{Q}{P_\mu} = 0
   \label{3.57}
\end{equation}
All members of a supermultiplet must have the same mass!

Only one structure of type (\ref{3.46c})
\begin{equation}
   \acomm{Q_a}{Q_b} = c_3(\gamma^\mu C )_{ab} P_\mu + c_4 (\Sigma^{\mu\nu} C)_{ab} M_{\mu\nu} \label{3.58}
\end{equation}
Note that LHS is symmetric under $a\leftrightarrow b$, hence need symmetric Dirac matrices on RHS: multiplication with $C$! Jacobi-identity
\begin{equation*}
   \acomm{Q_a}{\comm{Q_b}{P_\mu}} + \acomm{Q_b}{\comm{P_\mu}{Q_a}} + \comm{P_\mu}{\acomm{Q_a}{Q_b}} = 0 \quad \Rightarrow c_4 = 0
\end{equation*}
$c_3$ can be given any value by re-scaling $Q$, take
\begin{subequations}\label{3.59}
\begin{align}
   \acomm{Q_a}{Q_b} &= -2 (\gamma^\mu C)_{ab} P_\mu \label{3.59a} \\
   \acomm{Q_a}{\bar Q_b} &= 2 \gamma^\mu_{ab} P_\mu \label{3.59b} \\
   \acomm{\bar Q_a}{\bar Q_b} &= 2(C^{-1} \gamma^\mu)_{ab} P_\mu \label{3.59c}
\end{align} 
\end{subequations}
where the last two are from (\ref{3.59a}) and (\ref{3.50}).

(\ref{3.51}), (\ref{3.57}), and (\ref{3.59a}) are invariant under chiral rotation
\begin{equation}
   Q \rightarrow e^{i \chi \gamma_5} Q \label{3.60}
\end{equation}
with $\chi \in \R$.
It allows us to introduce another bosonic generator $R$ with
\begin{equation}
   \comm{Q_a}{R} = (\gamma_5)_{ab} Q_b \label{3.61}
\end{equation}
It leads to
\begin{equation}
   e^{-i \chi R} Q e^{i \chi R} = e^{i \chi \gamma_5} Q \label{3.62}
\end{equation}
It corresponds to axial global $\Uni(1)_R$ symmetry! Note that $i\gamma_5 Q$ is Majorana, $\gamma_5 Q$, and $i Q$ are not. 

In (\ref{3.59a}), LHS becomes for $|\chi| \ll 1$
\begin{align*}
   &\acomm{(\delta_{ac} + i \chi \gamma_{5 ac})Q_c}{(\delta_{bd} + i\chi \gamma_{5 bd})Q_d} \\
   &= \acomm{Q_a}{Q_b} + i \chi (\gamma_{5ac} \acomm{Q_c}{Q_b} + \gamma_{5bd} \acomm{Q_a}{Q_d}) + \order{\chi^2}  \\
   & \stackrel{!}{=} \acomm{Q_a}{Q_b}
\end{align*}
since $P_\mu \stackrel{R}{\rightarrow} P_\mu$. 
Hence
\begin{align*}
   \gamma_{5ac} (\gamma_\mu C)_{cb} + \gamma_{5 bd} (\gamma_\mu C)_{ad} \stackrel{!}{=} 0 \\
   \Rightarrow (\gamma_5 \gamma_\mu C)_{ab} + (\gamma_\mu C \gamma_5)_{ab} \stackrel{!}{=} 0
\end{align*}
since 
\begin{align*}
   C^{-1} \gamma_5 C \stackrel{(\ref{1.42})} = \gamma_5^T = \gamma_5 \\
   \Rightarrow \gamma_5 C = C \gamma_5, \quad \gamma_\mu \gamma_5 = - \gamma_5 \gamma_\mu
\end{align*}

\paragraph{SUSY algebra}
\begin{subequations}\label{3.63}
\begin{align}
   \acomm{Q_a}{Q_b} &= -2 (\gamma^\mu)_{ab} P_\mu \label{3.63a} \\
   \comm{Q_a}{P_\mu} &= 0 \label{3.63b}\\
   \comm{M_{\mu\nu}}{Q_a} &= -(\Sigma_{\mu\nu})_{ab} Q_b \label{3.63c} \\
   \comm{Q_a}{R} &= (\gamma_5)_{ab} Q_b \label{3.63d} \\
   \comm{R}{P_\mu} &= \comm{R}{M_{\mu\nu}} = 0 \label{3.63e}
\end{align}
\end{subequations}
Note that a single Majorana spinor supercharge $Q$: $N=1$ supersymmetry is introduced. For $N>1$, we have extended SUSY: have $N$ supercharges $Q^i$, $i=1,\dots, N$.

Upper index $i$ can be gauged and it leads to extra bosonic generators, hence extra terms in (\ref{3.63a}): central charges.

SUSY theories with $N>1$ have nice theoretical properties, e.g.~$N=4$ SYM are finite! However, these theories are not directly relevant for phenomenology, since they are not chiral! Thus we just ignore this option.

(\ref{3.63}) imply that each supersymmetric representation with fixed non-vanishing momentum must have equal number of bosonic and fermionic states.
Reason being that application of $P_\mu$ leaves number of states invariant. Application of $Q_a Q_b + Q_b Q_a$ leaves number of states invariant. Then application of single supercharge $Q$ leaves number of states invariant. Result follows from (\ref{3.29a}) and (\ref{3.29b}) (Detailed proof in homework or in \cite{drees_godbole_roy_2008}).

\paragraph{Remarks}
\begin{itemize}
   \item Equality of bosonic and fermionic degrees of freedom (d.o.f.~) holds both on- and off-shell.
   \item Result does not necessarily hold for the ground state, if
      \begin{equation*}
         P_\mu \ket{0} = 0
      \end{equation*}
      Difference ($n_B - n_F$) in ground state is called "Witten index"\cite{Witten:1982df}.
\end{itemize}

\paragraph{HLS theorem}
(Haag, Lopuszanski, Sohnius) is the most general symmetry of $S$-matrix of interacting QFT: (possibly extended) Super Poincaré algebra $\times$ internal symmetry, with supercharges transforming like spin-$1/2$ spinors under the homogeneous Lorentz group.

Since the bosonic symmetries in HLS theorem are used by nature. This can be viewed as theoretical argument in favor of SUSY.

\paragraph{Two-component spinors}
So far we have been writing supercharges as Majorana $4$-spinor. The description of the SUSY algebra, and in particular the construction of supersymmetric field theories, is much simpler using irreducible representations of homogeneous Lorentz group: $2$-component (Weyl) spinors.

$\xi_A$ with $A=1,2$ (Lorentz) transforms like $(\frac{1}{2},0)$
\begin{equation}
   \xi_A \rightarrow M_A^{\ B} \xi_B \label{3.64}
\end{equation}
with $M$ a complex $2 \times 2$ matrix an element of $\SL(2,\Co)$. $\SL(2,\Co)$ is the "universal covering group" of $\SO(1,3)$.

$\bar{\chi}_{\dot{A}}$ with $\dot{A} = 1,2$ transforms like $(0, \frac{1}{2})$
\begin{equation}
   \bar \chi_{\dot A} \rightarrow (M^*)_{\dot{A}}^{\ \dot{B}} \bar \chi_{\dot B} \label{3.65}
\end{equation}
Note the dotted indices are used for conjugated fields onward. 

Spinor indices can be raised or lowered by using rank-$2$ anti-symmetric tensor
\begin{equation}
   \epsilon^{12} = -\epsilon^{21} = - \epsilon_{12} = \epsilon_{21} = 1 \label{3.66}
\end{equation}
It has the following identities
\begin{subequations} \label{3.67}
\begin{align}
   \epsilon_{AB} \epsilon_{CD} &= \epsilon_{AC} \epsilon_{BD} - \epsilon_{AD} \epsilon_{BC} \label{3.67a} \\
   \epsilon^{AB} \epsilon_{CD} &= \delta^A_D \delta^B_C - \delta^A_C \delta^B_D \label{3.67b} \\
   \xi^A &= \epsilon^{AB} \xi_B \label{3.67c} \\
   \xi_A &= \epsilon_{AB} \xi^{B} \label{3.67d} \\
   \bar{\chi}^{\dot{A}} &= \epsilon^{\dot A \dot B} \bar \chi_{\dot{B}} \label{3.67e}\\
   \bar\chi_{\dot A} &= \epsilon_{\dot A \dot B} \bar \chi ^{\dot B} \label{3.67f}
\end{align}
\end{subequations}

Consistency check
\begin{equation*}
   \xi^A = \epsilon^{AB} \xi_B = \epsilon^{AB} \epsilon_{BC} \xi^C = (\delta^A_C \delta^B_B - \delta^A_B \delta^B_C) \xi^C = \xi^A
\end{equation*}
with 
\begin{equation}
   \epsilon^{AB} \epsilon_{BC} = \delta^A_C \label{3.67g}
\end{equation}

$(\frac{1}{2},0)$ can go from $(0, \frac{1}{2})$ through complex conjugation
\begin{alignat}{2}
   &\xi^\dagger = \bar\xi &&\Rightarrow \xi_A = (\bar \xi_{\dot A})^\dagger, \xi^A = (\bar \xi^{\dot A})^\dagger \label{3.68a} \\
   &\bar\chi^\dagger = \chi &&\Rightarrow \bar\chi_{\dot A} = (\chi_A)^\dagger, \bar \chi^{\dot A} = (\chi^A)^\dagger \label{3.68b}
\end{alignat}
Lorentz-invariant spinor contraction
\begin{subequations}
\label{3.69}
   \begin{align}
      \xi \chi &= \xi^A\chi_A \label{3.69a} \\
   \bar \chi \bar\xi &= \bar\chi_{\dot A} \bar\xi^{\dot A} \label{3.69b}
\end{align}
  
\end{subequations}

Note the ordering of indices, since
\begin{equation}
   \xi^A \chi_A = \epsilon^{AB} \epsilon_{AC} \xi_B \chi^C = - \xi_B \chi^B \label{3.70}
\end{equation}

Proof of Lorentz invariance of (\ref{3.69a})
\begin{align*}
   \xi^A \chi_A &\stackrel{(\ref{3.67c})}{=} \epsilon^{AB} \xi_B \chi_A \stackrel{(\ref{3.64})}{\rightarrow} \epsilon^{AB} M_B^{\ C} \xi_C M_A^{\ D} \chi_D \\
                &\stackrel{(\ref{3.67d})}{=} \left( \epsilon^{AB}M_B^{\ C} \epsilon_{CE} M_A^{\ D} \right) \xi^E \chi_D = \xi^D \chi_D = \xi \chi
\end{align*}
where the terms in bracket are
\begin{align*}
   M^T \epsilon^{\wedge} M \epsilon_{\vee} &= \begin{pmatrix} M_{11} & M_{21} \\ M_{12} & M_{22}\end{pmatrix} \begin{pmatrix} 0 & 1 \\ -1 & 0 \end{pmatrix} \begin{pmatrix} M_{11} & M_{12} \\ M_{21} & M_{22 }\end{pmatrix} \begin{pmatrix} 0 & -1 \\ 1 & 0 \end{pmatrix} \\
   & =\begin{pmatrix} -M_{21} & M_{11} \\ -M_{22} & M_{12} \end{pmatrix} \begin{pmatrix} M_{12} & -M_{11} \\ M_{22} & -M_{21} \end{pmatrix} \\
   &= \begin{pmatrix} M_{11} M_{22} - M_{12} M_{21} & M_{11} M_{21} - M_{11} M_{21} \\ -M_{12} M_{22} + M_{12} M_{22} & M_{11} M_{22} - M_{12} M_{21}\end{pmatrix} \\
   &= \begin{pmatrix} 1 & 0 \\ 0 & 1\end{pmatrix}
\end{align*}
since $\det M = 1$.

Generalized Pauli matrices connect $(\frac{1}{2}, 0)$ and $(0, \frac{1}{2})$ spinor, and they are Lorentz $4$-vector
\begin{subequations}
  \label{3.71}
  \begin{align}
     \xi \sigma^\mu \bar\chi &= \xi^A \sigma^\mu_{A \dot{B}} \bar\chi^{\dot{B}} \label{3.71a} \\
     \bar\chi \bar\sigma^{\mu} \xi &=  \bar\chi_{\dot A} \bar \sigma^{\mu \dot{A} B} \xi_B \label{3.71b}
  \end{align} 
\end{subequations}
Identities for $\sigma$ matrices in \cite{drees_id} (I1-I3) .

If spinor $\xi$, $\chi$ contain anti-commuting components (fermion field operator, Grassmann coordinates):
\begin{subequations}\label{3.72}
   \begin{align}
      \xi \chi &\stackrel{(\ref{3.69a})}{=} \xi^A \chi_A \stackrel{(\ref{3.70})}{=} - \xi_A \chi^A = \chi^A \xi_A = \chi\xi \label{3.72a} \\
      \bar \chi \bar \xi &= \bar \xi \bar\chi \label{3.72b}
   \end{align}
\end{subequations}
Introduce $2$-spinor of Grassmann coordinates $\theta_A$, $\bar \theta_{\dot{A}}$
\begin{subequations}
  \label{3.73}
  \begin{align}
     \theta \theta &\stackrel{(\ref{3.69a})}{=} \theta^A \theta_A \stackrel{(\ref{3.67})}{=} \epsilon^{AB} \theta_B \theta_A \stackrel{(\ref{3.66})}{=} +\theta_2 \theta_1 - \theta_1 \theta_2 = -2 \theta_1 \theta_2 \label{3.73a}\\
     \bar\theta \bar\theta &\stackrel{\eqref{3.69b}}{=} \bar \theta_{\dot{A}} \bar\theta^{\dot{A}} \stackrel{\eqref{3.67f}}{=} \epsilon_{\dot{A}\dot{B}} \bar\theta^{\dot{B}}\bar\theta^{\dot{A}} \stackrel{\eqref{3.66}}{=} -\bar\theta^{\dot{2}} \bar\theta^{\dot{1}} +\bar \theta^{\dot{1}} \bar\theta^{\dot{2}} = 2 \bar{\theta}^{\dot{1}} \bar \theta^{\dot{2}}\label{3.73b}
  \end{align} 
\end{subequations}
Don't confuse $\theta \theta$ (with $\theta$ being 2-spinor) with $\theta^2 (=0)$ (with $\theta$ being a Lorentz scalar)!

$2$-spinor identities in \cite{drees_id} (4-6).


\paragraph{Making $4$-spinors from $2$-spinors}
Dirac spinor contains two different $2$-spinors (in chiral representation)
\begin{subequations}
   \label{3.74}
   \begin{alignat}{2}
     &\psi = \begin{pmatrix} \xi_{\vee} \\ \bar\chi^{T \wedge} \end{pmatrix},  &&\psi_a= \begin{pmatrix} \xi_A \\ \bar\chi_{\dot{B}}\end{pmatrix} \label{3.74a} \\
     &\bar\psi = \psi^\dagger \gamma^0 = \begin{pmatrix} \bar\xi_\vee \\ \chi^{T \wedge}\end{pmatrix} \begin{pmatrix} 0 & \id \\ \id & 0 \end{pmatrix} = \begin{pmatrix} \chi^{T \wedge} & \bar \xi_\vee\end{pmatrix},\quad  &&\bar\psi_a = \begin{pmatrix} \chi^A & \bar\xi_{\dot{B}} \end{pmatrix} \label{3.74b} \\
     &\psi^C \stackrel{(\ref{1.42})}{=} C \bar\psi^T \stackrel{(\ref{3.74b})(\ref{3.75})}{=} \begin{pmatrix} \epsilon_\vee & 0 \\ 0 & \epsilon^\wedge \end{pmatrix} \begin{pmatrix} \chi^{\wedge} \\ \bar\xi^{T}_{\vee}\end{pmatrix} \stackrel{(\ref{3.67c})(\ref{3.67d})}{=} \begin{pmatrix} \chi_\vee \\ \bar\xi^{\wedge T}\end{pmatrix}  \label{3.74c}
  \end{alignat} 
\end{subequations}
In chiral representation, the charge conjugation matrix
\begin{equation}
   C = i \gamma^2 \gamma^0 \stackrel{(\ref{0.5})}{=} \begin{pmatrix} -i\sigma^2 & 0 \\ 0 & i \sigma^2 \end{pmatrix} = \begin{pmatrix} \epsilon_{\vee} & 0 \\ 0 & \epsilon^\wedge  \end{pmatrix} \label{3.75}
\end{equation}

Majorana spinor
\begin{equation}
   \lambda_M = \begin{pmatrix} \xi_\vee \\ \bar \xi^{T \wedge} \end{pmatrix} \label{3.76}
\end{equation}
contains only $2$ d.o.f..
With (\ref{3.74c}) 
\begin{equation*}
   \lambda_M^C = \begin{pmatrix} \xi_\vee \\ \bar\xi^{T\wedge}\end{pmatrix} = \lambda_M
\end{equation*}

Identities with $4$-spinor in \cite{drees_id} (11-13).

In (\ref{3.50}), we had introduced Majorana $4$-spinor of supercharges
\begin{equation}
   Q_a = \begin{pmatrix} Q_A \\ {\bar{Q}}^{\dot{B}} \end{pmatrix} \label{3.77}
\end{equation}
where the entries individually are $2$-spinors of supercharge. 

In terms of these, (\ref{3.63}) become
\begin{subequations}
   \label{3.78}
  \begin{align}
  \acomm{Q_A}{{\bar{Q}}_{\dot{B}}} &= 2 \sigma^\mu_{A \dot{B}} P_\mu  \Rightarrow \acomm{ {\bar{Q}}^{\dot{A}} }{Q^B} = 2 {\bar{\sigma}}^{\mu \dot{A} B} P_\mu \label{3.78a} \\
  \acomm{Q_A}{Q_B} &= \acomm{{\bar{Q}}_{\dot{A}}}{{\bar Q}_{\dot{A}}} = 0 \label{3.78b} \\
  \comm{Q_A}{P_\mu} &= \comm{{\bar{Q}}_{\dot{A}}}{P_\mu} = 0 \label{3.78c} \\
     \comm{M_{\mu\nu}}{Q_A}  &= - (\sigma_{\mu\nu})_A^{\ B} Q_B \label{3.78d} \\
     \comm{M_{\mu\nu}}{{\bar{Q}}^{\dot{A}}} &= - ({\bar{\sigma}}_{\mu\nu})^{\dot{A}}_{\ \dot{B}} \bar{Q}^{\dot{B}} \label{3.78e} \\
     \comm{Q_A}{R} &= Q_A \label{3.78f} \\
     \comm{{\bar{Q}}^{\dot{A}}}{R} &= - {\bar{Q}}^{\dot{A}} \label{3.78g}
   \end{align} 
\end{subequations}

\paragraph{Particle supermultiplets}
\begin{equation*}
   (\ref{3.78c}) = \comm{P^2}{Q} = \comm{P^2}{\bar Q} = 0 \label{3.79}
\end{equation*}
means that all members of a supermultiplet must have the same mass ! Thus SUSY must be broken!! $\tilde e$ doesn't exist with $m_{\tilde{e}} = m_e = \SI{511}{\kilo\eV}$.

To construct spin stats, consider
\begin{subequations}
   \begin{align}
   \comm{J^p}{Q_A} &\stackrel{(\ref{3.47})}{=} \frac{1}{2} \epsilon^{p r s} \comm{M_{rs}}{Q_A} \notag \\
                   &\stackrel{(\ref{3.78d})}{=} - \frac{1}{2} \epsilon^{prs} (\sigma_{rs})^{\ B}_A Q_B \notag \\
                   &\stackrel{(\ref{3.36b})}{=} - \frac{i}{8} \epsilon^{prs} (\sigma_r \bar \sigma_s - \sigma_s \bar\sigma_r)^{\ B}_{A} Q_B \notag \\
                   &\stackrel{(\ref{3.37})}{=} \frac{i}{8} \epsilon^{prs} (\sigma_r \sigma_s - \sigma_s \sigma_r)^{\ B}_A Q_B \notag \\ 
                   &= \frac{i}{8} \epsilon^{prs} (2i\epsilon_{rst}\sigma_t)_A^{\ B} Q_B  \notag \\
                   &= - \frac{1}{2} (\sigma^p)_A^{\ B} Q_B \label{3.80a} \\
   \comm{J^p}{\bar Q^{\dot{A}}} &= - \frac{1}{2} (\bar \sigma^p)^{\dot{A}}_{\ \dot{B}} \bar Q^{\dot{B}} \label{3.80b}
\end{align}
\end{subequations}

Consider massless superfield, $P^2 = 0$. Go to frame where $P_\mu = \omega (1,0,0,1)$ and $J^3$ measures helicity.
\begin{equation}
   \sigma^\mu P_\mu = \omega (\id_{2\times 2} - \sigma_3) = \omega \begin{pmatrix} 0 & 0 \\ 0 & 2 \end{pmatrix}
\end{equation}
thus from (\ref{3.78})
\begin{subequations}
   \label{3.81}
  \begin{align}
     \acomm{Q_1}{\bar{Q}_{\dot{1}}} &= 0 \label{3.81a} \\
     \acomm{Q_2}{\bar{Q}_{\dot{2}}} &= 4 \omega \label{3.81b} \\ 
     \acomm{Q_1}{\bar{Q}_{\dot{2}}} &= \acomm{Q_2}{\bar{Q}_{\dot{1}}} = 0 \label{3.81c}
  \end{align} 
\end{subequations}
Since $\bar{Q}_1$ is conjugate of $Q_1$, and Hilbert state only include states with positive norm, (\ref{3.81a}) implies $Q_1 = \bar{Q}_{\dot{1}} = 0$ within this space.

Define 
\begin{equation}
   Q = \frac{\bar{Q}_{\dot{2}}}{2\sqrt{\omega}}, \quad \bar{Q} = \frac{Q_2}{2 \sqrt{\omega}} \label{3.82}
\end{equation}
From (\ref{3.81b})
\begin{equation}
   \acomm{Q}{\bar{Q}} = 1;\quad \acomm{Q}{Q} = \acomm{\bar{Q}}{\bar{Q}} = 0 \label{3.83}
\end{equation}
From (\ref{3.80a})
\begin{subequations}
   \label{3.84}
  \begin{align}
     \comm{J^3}{Q_2} &= - \frac{1}{2} (\sigma_3)^{\ B}_2 Q_B = \frac{1}{2} Q_2 \label{3.84a} \\
     \comm{J^3}{\bar{Q}_{\dot{2}}} &= - \frac{1}{2} \bar{Q}_{\dot{2}} \label{3.84b}
  \end{align} 
\end{subequations}

Let $\ket{j_3}$ be eigenstate if $J_3$
\begin{equation}
   J_3 \ket{j_3} = j_3 \ket{j_3} \label{3.85}
\end{equation}

\begin{subequations}
   \label{3.86}
  \begin{align}
     J_3 Q \ket{j_3} &\stackrel{(\ref{3.82})}{=} \frac{1}{2 \sqrt{\omega}} J_3 \bar{Q}_{\dot{2}} \ket{j_3} \stackrel{(\ref{3.84b})}{=} \frac{1}{2\sqrt{\omega}} (\bar{Q}_{\dot{2}}J_3 - \frac{1}{2} \bar{Q}_{\dot{2}}) \ket{j_3} \notag \\
                     & \stackrel{(\ref{3.85}), (\ref{3.82})}{=} Q (j_3 - \frac{1}{2}) \ket{j_3} = (j_3 - \frac{1}{2}) Q \ket{j_3} \label{3.86a} \\
     J_3 \bar{Q} \ket{j_3} &= (j_3 + \frac{1}{2}) \bar{Q} \ket{j_3} \label{3.86b}
  \end{align} 
\end{subequations}
Application of $Q (\bar{Q})$ lowers (raises) $j_3$ by $\frac{1}{2}$ unit! Since $\bar{Q} \bar{Q} =0$, we must have state $\ket{j_\text{max}}$ with
\begin{equation}
   \bar{Q} \ket{j_\text{max}} = 0 \label{3.87}
\end{equation}
Then either $\bar{Q} \ket{j_0} = 0$ with $j_0 = j_\text{max}$ or $\bar{Q}\bar{Q} \ket{j_J} = 0$ with $j_J + \frac{1}{2} = j_\text{max}$.

Define
\begin{equation}
   \ket{j_\text{max} - \frac{1}{2}} = Q \ket{j_\text{max}} \label{3.88}
\end{equation}
then
\begin{equation*}
   Q \ket{j_\text{max} - \frac{1}{2}} = 0, \quad J_3 \ket{j_\text{max} - \frac{1}{2}} \stackrel{(\ref{3.85})}{=} (j_\text{max} - \frac{1}{2}) \ket{j_\text{max} - \frac{1}{2}}
\end{equation*}
Complete supermultiplet consists of $2$ states
\begin{equation}
   \ket{j_\text{max}}, \ket{j_\text{max} - \frac{1}{2}}
\end{equation}

CPT: for any state $\ket{j_3}$ transforming like $R$ under some internal symmetries, must exist state $\ket{-j_3}$ transforming like $\bar{R}$ (conjugate of $R$). Then there are two cases
\begin{itemize}
   \item $R\neq \bar{R}$ (e.g.~SM fermions): simplest irreducible representation of SUSY algebra has one Weyl fermion (or, equivalently, one helicity state of a Dirac fermion), and one complex scalar e.g.~$n_L$, $\tilde{n}_L$; or $n_R$, $\tilde{n}_R$: chral supermultiplets \label{3.90}
\item $R=\bar{R}$ (self-conjugate): simplest irrep. is \textit{real}, but contains both helicity state, e.g.~

   vector superfield: one vector (gauge) boson and one Majorana fermion ($2$ d.o.f.~each)
   \begin{equation}
   \text{hel.}_b = \pm 1, \quad \text{hel.}_f = \pm \frac{1}{2} 
   \label{3.91} .
   \end{equation}

   Gravity superfiel: one tensor graviton and one Majorana gravitino 
   \begin{equation}
      \text{hel}_g = \pm 2, \quad \text{hel.}_{\tilde{g}} = \pm \frac{3}{2} 
   \label{3.92}
   \end{equation}
\end{itemize}

\section{Free superfields} 
Fields are functions of spacetime $x^\mu = (t, \pmb{x})$.
Superfields are functions of superspace coordinates
\begin{equation}
   z = (x^\mu, \theta^A, \bar{\theta}_{\dot{A}})    \label{3.93}
\end{equation}
It include $4$ Grassmann variables. One can do calculus with these Grassmann variables: generalization of (\ref{3.21}) to (\ref{3.27}) to $2 \theta^A$, $2 \bar{\theta}^{\dot{A}}$ (complex conjugate to each other): $(I.16)-(I.24)$ in \cite{drees_id}.
Since $\theta_A^2 = \bar{\theta}_{\dot{A}}^2 = 0$, expand a superfield, generalizing (\ref{3.22})
\begin{equation}
   F(z) = f(x) + \sqrt{2} \theta \xi(x) + \sqrt{2} \bar{\theta} \bar{\chi}(x) + \theta \theta M(x) + \bar{\theta} \bar{\theta} N(x) + \theta \sigma^\mu \bar{\theta}A_\mu (x) + \theta \theta \bar{\theta} \bar{\lambda}(x) + \bar{\theta}\bar{\theta} \theta \zeta (x) + \frac{1}{2} \theta\theta\bar{\theta}\bar{\theta} D(x) \label{3.94}
\end{equation}

Explicit representation of momentum generator
\begin{equation}
   P_\mu = i \pdv{x^\mu} = i \partial_\mu \label{3.95}
\end{equation}
Similarly, explicit representation of SUSY generator
\begin{subequations}
   \label{3.96}
  \begin{align}
     Q_A &= -i(\partial_A + i \sigma^\mu_{A \dot{B}} \bar{\theta}^{\dot{B}} \partial_\mu) \label{3.96a} \\
     \bar{Q}^{\dot{A}} &= -i(\bar{\partial}^{\dot{A}} + i \theta^B \sigma^\mu_{B \dot{C}} \epsilon^{\dot{C}\dot{A}} \partial_\mu) \label{3.96b}
  \end{align} 
\end{subequations}
with the derivative $\partial_A = \partial/\partial \theta^A$ and $i \theta^B \sigma^\mu_{B \dot{C}} \epsilon^{\dot{C} \dot{A}} \partial_\mu \stackrel{(I2b)}{=} i \bar{\sigma}^{\mu \dot{A} B} \theta_B \partial_\mu$.

To check the consistency, first
\begin{align*}
   \bar{Q}_{\dot{B}} &= \epsilon_{\dot{B}\dot{A}} \bar{Q}^{\dot{A}} \\
                     &= -i \left( \epsilon_{\dot{B}\dot{A}} \bar\partial^{\dot{A}} + i \epsilon_{\dot{B}\dot{A}} \epsilon^{\dot{C}\dot{A}} \theta^D \sigma^\mu_{D \dot{C}} \partial_\mu \right) \\
                     & \stackrel{(I.17) (\ref{3.67g})}{=} -i (- \bar{\partial}_{\dot{B}} - i \delta^{\dot{C}}_{\dot{B}} \theta^D \sigma^\mu_{D \dot{C}} \partial_\mu) \\
   \bar{Q}_{\dot{B}} &= i (\bar{\partial}_{\dot{B}} + i \theta^D \sigma^\mu_{D \dot{B}} \partial_\mu)
\end{align*}

Recall $\theta^A \stackrel{(\ref{3.67c})}{=} \epsilon^{AB}\theta_B$
\begin{equation*}
   \theta^1 = \theta_2, \theta^2 = -\theta_1
\end{equation*}
Then
\begin{equation*}
   \epsilon^{AB}\partial_B = \epsilon^{AB} \pdv{\theta^B} = 
   \begin{cases}
      A = 1 & \pdv{\theta^2} = - \pdv{\theta_1}  \\
      A = 2 & -\pdv{\theta^1} = - \pdv{\theta_2}
   \end{cases} 
   = - \partial^A (\Rightarrow (I.17))
\end{equation*}

It is "obvious" that
\begin{equation*}
   \acomm{Q_A}{Q_B} = \acomm{\bar{Q}_{\dot{A}}}{\bar{Q}_{\dot{B}}} = 0
\end{equation*}
since
\begin{equation*}
   \acomm{\partial_A}{\partial_B} = \acomm{\partial_A}{\bar{\theta}^{\dot{B}}} = \acomm{\bar{\theta}^{\dot{A}}}{\bar{\theta}^{\dot{B}}} = 0
\end{equation*}
And
\begin{align*}
   \acomm{Q_A}{\bar{Q}_{\dot{B}}} &= \acomm{\partial_A + i\sigma^\mu_{A\dot{C}}\bar{\theta}^{\dot{C}}\partial_\mu}{\bar\partial_{\dot{B}} + i \theta^D \sigma^\nu_{D\dot{B}}\partial_\nu} \\
                                  &= i \delta^D_A \sigma^\nu_{D\dot{B}} \partial_\nu + i\delta^{\dot{C}}_{\dot{B}} \sigma^\mu_{A\dot{C}}\partial_\mu \\
                                  &= 2 i \sigma^\mu_{A\dot{B}} \partial_\mu  \\
                                  &\stackrel{(\ref{3.95})}{=} 2 \sigma^\mu_{A\dot{B}} P^\mu
\end{align*}

Coordinate shift generated by $P_\mu$
\begin{equation*}
   f(x+\delta) = f(x) + \delta^\mu \partial_\mu f(x) + \order{\delta^2} \Rightarrow \delta f = i \delta \cdot \underline{P} f
\end{equation*}
Analogously SUSY transformation is a shift in superspace!
\begin{subequations}
   \label{3.97}
  \begin{align}
     F &\rightarrow F + \delta F \label{3.97a} \\
     \delta F &= i (\epsilon Q + \bar{\epsilon} \bar{Q}) F \label{3.97b}
  \end{align} 
\end{subequations}
is a global SUSY transformation with $\epsilon$ and $\bar{\epsilon}$ infinitesimal Grassmann variables, independent of $x$.
\begin{subequations}
   \label{3.98}
  \begin{align}
     (\ref{3.78a}) &\Rightarrow [Q] = [E]^{1/2} \label{3.98a} \\
     (\ref{3.96}) &\Rightarrow [\theta] = [\epsilon] = [E]^{-1/2} \label{3.98b}
  \end{align} 
\end{subequations}
Equivalently
\begin{equation}
   \delta F = F(x^\mu - i \theta \sigma^{\mu}\bar \epsilon + i \epsilon \sigma^\mu \bar\theta, \theta + \epsilon, \bar\theta + \bar\epsilon) - F(x, \theta, \bar\theta) \label{3.99}
\end{equation}

Taylor-expand this around $(x,\theta,\bar\theta)$ to first order in $\epsilon$, $\bar\epsilon$, compare with original F and read off component fields! 

Terms in (\ref{3.94}) without $\theta$, $\bar\theta$
\begin{subequations}
   \label{3.100}
   \begin{align}
      \delta f(x) &= \sqrt{2} \epsilon \xi (x) + \sqrt{2} \bar{\epsilon} \bar{\chi}(x) \label{3.100a}
      \shortintertext{Terms with one $\theta$, no $\bar\theta$}
      \delta (\sqrt{2} \theta \xi (x)) &\stackrel{(\ref{3.72})}{=}  (\epsilon \theta + \theta \epsilon) M(x) + (\partial_\mu f(x)) \cdot (-i \theta \sigma^\mu \bar\epsilon) + \theta \sigma^\mu \bar\epsilon A_\mu (x) \notag \\
      \delta \xi_A(x) &= \sqrt{2} \epsilon_A M(x) + \frac{1}{\sqrt{2}} (\sigma^\mu \bar\xi)_A \left[ -i\partial_\mu f(x) + A_\mu (x) \right] \label{3.100b}
      \shortintertext{Similarly,}
      \delta \bar{\xi}^{\dot{A}} &= \sqrt{2} \bar\xi^{\dot{A}} N - (\bar\sigma^\mu \epsilon)^{\dot{A}} (i\partial_\mu f + A_\mu) \label{3.100c} \\
      \delta M &= \bar\epsilon \bar\lambda + \frac{i}{\sqrt{2}} \partial_\mu \xi \sigma^\mu \bar\epsilon \label{3.100d} \\
      \delta N &= \epsilon \zeta - \frac{i}{\sqrt{2}} \epsilon \sigma^\mu \partial_\mu \bar\chi \label{3.100e} \\
      \delta A_\mu &= \epsilon \sigma_{\mu} \bar\lambda + \zeta \sigma_\mu \bar\epsilon - \frac{i}{\sqrt{2}} \epsilon \partial_\mu \xi + \frac{i}{\sqrt{2}} \partial_\mu \bar\chi \bar\epsilon + i\sqrt{2} \epsilon \sigma_{\mu\nu} \partial^\nu \xi - i\sqrt{2} \bar\epsilon \bar\sigma_{\mu\nu} \partial^\nu \bar\chi \label{3.100f} \\
      \delta \bar\lambda^{\dot{A}} &= \bar\epsilon^{\dot{A}} D - \frac{i}{2} \bar\epsilon^{\dot{A}} \partial^\mu A_\mu - i (\bar\sigma^\mu \epsilon)^{\dot{A}} \partial_\mu M + (\bar\sigma^{\mu\nu}\bar\epsilon)^{\dot{A}}\partial_\mu A_\nu \label{3.100g} \\
      \delta \xi_A &= \epsilon_A D + \frac{i}{2} \epsilon_A \partial^\mu A_\mu - i (\sigma^\mu \bar\epsilon)_A \partial_\mu N - (\sigma^{\mu\nu} \epsilon)_A \partial_\mu A_\nu \label{3.100h} \\
      \delta D &= i\partial_\mu (\zeta \sigma^\mu \bar\epsilon + \bar\lambda \bar\sigma^\mu \epsilon) \label{3.100i}
   \end{align} 
\end{subequations}
Here $f(x)$, $M(x)$, $N(x)$ and $D(x)$ are scalar fields, $A_\mu(x)$ vector field, $\xi_A(x)$, and $\chi_A(x)$ two left-handed Weyl spinor fields and $\bar\chi^{\dot{A}}(x)$, and $\bar{\lambda}^{\dot{A}}$ two right-handed Weyl spinor fields.


Remarks
\begin{itemize}
   \item bosonic fields transform into fermionic ones and vice versa
   \item $D$ transforms into total spacetime derivative $\Rightarrow F|_D = \int \dd[4]{\theta} F$ appearing in Lagrangian gives SUSY invariant action (ignoring surface terms)
   \item (\ref{3.96}) are linear operators, thus linear combinations of superfields are superfields
   \item Expansion (\ref{3.94}) is completely general, thus products of superfields are also superfields
   \item SUSY algebra closes on (\ref{3.94}) (of course,) but this representation is reducible, i.e.~closure can be achieved with fewer component fields
\end{itemize}

We had seen
\begin{itemize}
   \item Explicit form of SUSY generator contains derivatives with respect to Grassmannian and spacetime coordinates.
   \item SUSY transformation is a translation in superspace: allows to read off transformations of component fields of most general (scalar) superfield.
   \item This most general superfield is a \textit{reducible} representation of SUSY algebra.
\end{itemize}

\paragraph{Irreducible representation}
We can construct irreducible representation with the help of chiral SUSY covariant derivatives.
Note that $\acomm{Q_A}{{\partial}_A} \neq 0$ and $\acomm{\bar{Q}_{\dot{A}}}{\partial_{\dot{A}}} = 0$, thus $\partial_A$ and $\bar\partial_{\dot{A}}$ are not SUSY covariant.

Define covariant derivatives
\begin{subequations}
   \label{3.101}
   \begin{align}
      \D_A &= \partial_A -i \sigma^{\mu}_{A\dot{B}} \bar\theta^{\dot{B}} \partial_\mu \label{3.101a} \\
      \bar{\D}_{\dot{A}} &= - \partial_{\dot{A}} + i \theta^B \sigma^\mu_{B\dot{A}} \partial_\mu \label{3.101b}
   \end{align} 
\end{subequations}

They commute with supercharge generators
\begin{subequations}
\label{3.102}   
\begin{align}
   \acomm{\D_A}{Q_B} & \stackrel{(\ref{3.96a})(\ref{3.101a})}{=} -i \acomm{\partial_A - i \sigma^\mu_{A\dot{B}} \bar\theta^{\dot{B}}\partial_\mu}{\partial_B + i\sigma^\nu_{B\dot{C}}\bar\theta^{\dot{C}}\partial_\nu} = 0 \label{3.102a} \\
         \acomm{\D_A}{\bar{Q}_{\dot{B}}} &\stackrel{(\ref{3.96b}) (\ref{3.100a})}{=} i \acomm{\partial_A - i \sigma^\mu_{A\dot{C}}\bar\theta^{\dot{C}}\partial_\mu}{\bar\partial_{\dot{B}} + i\theta^D \sigma^\nu_{D\dot{B}} \partial_\nu} \notag \\
                                      & \stackrel{(I.16a,c)}{=} i \left( i\delta^D_A\sigma^\nu_{D\dot{B}} \partial_\nu - i \sigma^\mu_{A\dot{C}} \delta^{\dot{C}}_{\dot{B}} \partial_\mu \right) = 0\label{3.102b}
\end{align}
\end{subequations}

Contravariant version
\begin{subequations}
   \label{3.103}
\begin{align}
   \D^A &= \epsilon^{AB}\D_B \stackrel{(I.17a)}{=} - \partial^A - i \epsilon^{AB} \sigma^{\mu}_{B\dot{B}} \bar\theta^{\dot{B}} \partial_\mu \notag \\
           & \stackrel{(\ref{3.67e})}{=} -\partial^A - i \epsilon^{AB} \sigma^\mu_{B\dot{B}} \epsilon^{\dot{B} \dot{C}}\bar{\theta}_{\dot{C}}\partial_\mu \notag \\
           & \stackrel{(I.2a)}{=} -\partial^A + i \bar\theta_{\dot{C}} \bar\sigma^{\mu \dot{C}A}\partial_\mu \label{3.103a} \\
         \bar\D^{\dot{A}} &= \epsilon^{\dot{A}\dot{B}}\bar\D_{\dot{B}} = \bar\partial^{\dot{A}} - i\bar\sigma^{\mu \dot{A} B} \theta_B \partial_\mu \label{3.103b}
\end{align}
\end{subequations}

Product of two covariant derivative (index $A$ not summed!)
\begin{subequations}
  \begin{align}
     \D_A \D_A &\stackrel{\ref{3.101a}}{=} \left(  \partial_A - i \sigma^\mu_{A\dot{B}} \bar\theta^{\dot{B}} \partial_\mu \right) \left( \partial_A - i \sigma^\nu_{A\dot{C}} \bar\theta^{\dot{C}}\partial_\nu \right) \notag \\
               &= \partial_A \partial_A -i\sigma^\mu_{A\dot{B}} \partial_A\partial_A - \partial_A i \sigma^\nu_{A\dot{C}} \bar{\theta}^{\dot{C}} \partial_\nu - \sigma^\mu_{A\dot{B}} \bar{\theta}^{\dot{B}} \partial_\mu \sigma^\nu_{A\dot{C}} \bar\theta^{\dot{C}} \partial_\nu \notag
               \shortintertext{since $\acomm{\partial_A}{\partial_A} = \acomm{\partial_A}{\bar\theta^{\dot{B}}} = 0$}
               & \stackrel{(I.7f)}{=} -\sigma^\mu_{A\dot{B}} \sigma^\nu_{A\dot{C}} \frac{1}{2} \epsilon^{\dot{B}\dot{C}} \bar\theta \bar\theta \partial_\mu\partial_\nu \notag \\
               & = - (\sigma \cdot \partial)_{A\dot{B}} (\sigma \dot \partial)_{A\dot{C}} \cdot \frac{1}{2} \epsilon^{\dot{A}\dot{B}} \bar\theta \bar\theta = 0 \label{3.104a} \\
     \bar\D_{\dot{A}} \bar\D_{\dot{A}} &= 0 \label{3.104b}
   \end{align} 
\end{subequations}
In this regard, SUSY covariant derivatives are like Grassmann derivatives
\begin{subequations}
   \label{3.105}
  \begin{align}
     \D_A \D_B &= \frac{1}{2} \epsilon_{AB}\D \D = \frac{1}{2} \epsilon_{AB} \D^C\D_C \label{3.105a} \\
     \bar\D_{\dot{A}} \bar\D_{\dot{B}} &= - \frac{1}{2} \epsilon_{\dot{A}\dot{B}} \bar\D\bar\D = - \frac{1}{2} \epsilon_{\dot{A}\dot{B}} \bar\D_{\dot{C}} \bar\D^{\dot{C}} \label{3.105b} \\
     \D_A \D_B \D_C &= \bar\D_{\dot{A}} \bar\D_{\dot{B}}\bar\D_{\dot{C}} = 0 \label{3.105c}
  \end{align} 
\end{subequations}

Anti-commutators of SUSY covariant derivatives
\begin{subequations}
   \begin{align}
      \acomm{\D_A}{\D_B} &= \acomm{\bar\D_{\dot{A}}}{\bar\D_{\dot{B}}} = 0 \label{3.106a} \\
      \acomm{\D_A}{\bar\D_{\dot{B}}} &= 2i \sigma^\mu_{A\dot{B}} \partial_\mu \label{3.106b} \\
      \acomm{\D^A}{\bar\D^{\dot{B}}} &= 2i \bar\sigma^{\mu\dot{B}A} \partial_\mu \label{3.106c}
   \end{align} 
\end{subequations}
i.e.~$\acomm{\D}{\bar\D}$ satisfy \textit{same} anti-commutation relation as $\acomm{Q}{\bar Q}$! Also it carries dimension $[\text{energy}]^{1/2}$.

A \textit{left-chiral} superfield $\Phi$ and a \textit{right-chiral} $\Phi^\dagger$ are individually  defined by
\begin{equation}
   \bar\D_{\dot{A}} \Phi = 0,\; \quad \D_A \Phi^\dagger = 0 \label{3.108}
\end{equation}

These constraints are most easily implemented by defining left- and right-chiral superspace coordinates
\begin{equation}
   y^\mu = x^\mu - i \theta \sigma^\mu \bar\theta;\quad \bar{y}^{\mu} = x^\mu + i\theta \sigma^\mu \bar\theta \label{3.109}
\end{equation}
It satisfies
\begin{equation}
   \bar{\D}_{\dot{A}} y^\mu = \D_A \bar{y}^\mu = 0\label{3.110}
\end{equation}
Proof 
\begin{align*}
   \bar{\D}_{\dot{A}} y^\mu &\stackrel{(\ref{3.101b}), (\ref{3.109})}{=} \left( - \bar\partial_{\dot{A}} + i \theta^B \sigma^\nu_{B\dot{A}}\partial_\nu \right) \left( x^\mu - i\theta^C\sigma^\mu_{C\dot{D}} \bar\theta^{\dot{D}} \right) \\
                            &= - i \theta^C \sigma_{\mu C \dot{A}} + i \theta^B \sigma^\mu_{B\dot{A}} = 0
\end{align*}
$\bar\D_{\dot{A}} \theta_B = 0$ is trivial.

Hence  (\ref{3.110}) implies
\begin{equation}
   \bar\D_{\dot{A}} f(y,\theta) = \D_A f^*(\bar{y},\bar{\theta}) = 0 \label{3.111}
\end{equation}
Note that $f(y,\theta)$ has no explicit $\bar\theta$ dependence and $\bar \D_{\dot{A}}$ (as its index suggests) contains only derivative with respect to $\bar\theta$.

It can also be seen by applying the chain rule
\begin{align*}
   &\partial_A f(y,\theta) = \partial_A f(x, \theta,\bar\theta) + (\partial_A y) \pdv{f}{x} \\
   &\Rightarrow \partial_A^{(y)} = \partial_A - i \sigma^\mu_{A\dot{B}} \bar\theta^{\dot{B}} \partial_\mu^{(y)}; \quad \bar\partial_{\dot{A}}^{(\bar{y})} = \bar{\partial}_{\dot{A}} - i \theta^B \sigma^\mu_{B \dot{A}} \partial_\mu^{(y)}
\end{align*}
Hence, from (\ref{3.101})
\begin{equation}
   \D_A^{(y)} = \partial_A - 2 i \sigma^\mu_{A\dot{B}} \bar\theta^{\dot{B}} \partial_\mu^{(y)};\quad \bar\D_{\dot{A}}^{(y)} = - \bar\partial_{\dot{A}} \label{3.112}
\end{equation}
Similarly
\begin{equation}
   \D_A^{(\bar{y})} = \partial_A ; \quad \bar{\D}_{\dot{A}}^{(\bar{y})} = - \bar\partial_{\dot{A}} + 2i \theta^B \sigma^\mu_{B\dot{A}} \partial_\mu^{(\bar{y})}
\end{equation}

Construction of chiral superfields is now almost trivial: take expansion like (\ref{3.94}), with $x \rightarrow y$ and dropping all $\bar\theta$ terms for left-chiral field
\begin{subequations}
\label{3.114}
\begin{align}
   \Phi (y, \theta) &= \phi(y) + \sqrt{2} \theta \xi(y) + \theta\theta F(y) \label{3.114a} \\
   \Phi^\dagger (\bar{y},\bar{\theta}) &= \phi^*(\bar{y}) + \sqrt{2} \bar\theta \bar\xi (\bar{y}) + \bar\theta \bar\theta F^*(\bar{y}) \label{3.114b}
\end{align}
\end{subequations}
They are left- and right-chiral individually.

Remarks
\begin{itemize}
   \item Want to use these to describe matter, thus $\xi$ should be physical fermion fields.
      \begin{equation*}
         [\xi] = [E]^{3/2}, \quad [\Phi] = [\phi] = [E]^{1}
      \end{equation*}
      can be physical spinor and scalar fields. But what about $[F] = [E]^2$?
   \item Degrees of freedom
      \begin{itemize}
         \item off-shell: $\phi, F$ are $2$ complex scalar: $4$ d.o.f. (bosons) and $\xi$ is complex $2$-component spinor: $4$ fermionic d.o.f. They match with each other!
         \item on-shell: $\xi$: $2$ fermion d.o.f (e.o.m is the first order in time derivative) and complex scalar still counts as $2$ d.o.f.. Thus we must get rid of $F$!
      \end{itemize}
\end{itemize}
We will see later that $F$ is auxiliary field.

SUSY transformation (\ref{3.100}) was written in $x$-space, not in $y$-space. Thus need (\ref{3.114}) in $x$-space! Insert (\ref{3.109}) and expand
\begin{subequations}
   \label{3.115}
\begin{align}
   \Phi (x,\theta, \bar\theta) &= \phi(x) - i\theta \sigma^\mu \bar\theta \partial_\mu \phi(x) - \frac{1}{4} \theta\theta \bar\theta \bar\theta \partial_\mu \partial^\mu \phi(x) + \sqrt{2} \theta \xi (x) + \frac{i}{\sqrt{2}} \theta\theta \partial_\mu \xi(x) \sigma^\mu \bar\theta + \theta\theta F(x) \label{3.115a} \\
   \Phi^\dagger (x,\theta,\bar\theta) &= \phi^*(x) + i\theta \sigma^\mu \bar\theta \partial_\mu \phi^*(x) - \frac{1}{4} \theta\theta\bar\theta\bar\theta \partial_\mu\partial^\mu \phi^*(x) + \sqrt{2} \bar\theta \bar\xi(x) - \frac{i}{\sqrt{2}} \bar\theta \bar\theta \theta \sigma^\mu \partial_\mu \bar\xi(x) + \bar\theta\bar\theta F^*(x) \label{3.115b}
\end{align}
\end{subequations}

Comparison with (\ref{3.94}) yields
\begin{align*}
   &f(x) = \phi(x);\quad \xi = \xi;\quad \bar\chi = 0;\quad M = F;\quad N=0;\quad A_\mu = -i\partial_\mu \phi \\
   &\bar\lambda = \frac{i}{\sqrt{2}} \partial_\mu \xi\sigma^\mu;\quad \xi=0;\quad D = -\frac{1}{2} \partial_\mu\partial^\mu \phi
\end{align*}
They are not independent! Hence
\begin{subequations}
   \label{3.116}
\begin{align}
   \delta f(x) &\stackrel{(\ref{3.100a})}{=} \delta \phi (x) = \sqrt{2} \epsilon \xi (x) \label{3.116a} \\
   \delta \xi_A(x) &\stackrel{(\ref{3.100b})}{=} \sqrt{2} \epsilon_A F(x) + \frac{1}{\sqrt{2}} (\sigma^\mu \bar\epsilon)_A \cdot (2(-i)\partial_\mu \phi(x)) \notag \\
                   &= \sqrt{2} \left[ \epsilon_A F(x) - i (\sigma^\mu \bar\epsilon)_A \partial_\mu \phi(x) \right] \label{3.116b} \\
   \delta F(x) &\stackrel{(\ref{3.100d})}{=} \frac{i}{\sqrt{2}} \partial_\mu \xi(x) \sigma^\mu \bar\epsilon \cdot 2 \label{3.116c} \\
   \delta N &= \delta \bar{\chi} = \delta \xi = 0 \label{3.116d}
\end{align}
\end{subequations}
(\ref{3.100f}, \ref{3.100g}, \ref{3.100i}) are also consistent with translation table!

Remarks
\begin{itemize}
   \item can also derive (\ref{3.116}) by re-writing $Q$, $\bar Q$ in $y$-space: analogous to $\D, \bar\D \rightarrow \D^{(y)}, \bar\D^{(\bar{y})}$ !
   \item (\ref{3.114a}) leads to any product of left-chiral superfields is also a left-chiral superfield (no $\bar\theta$ dependence when written in $y$-space).

      (\ref{3.114b}) leads to any product of right-chiral superfields is also a right-chiral superfield.

      But product of a left- and a right-chiral is neither left- nor right-chiral!
\end{itemize}

Highest component of left-chiral superfield transforms into total derivative. Thus $\int \dd[N]{\theta} \prod_{i=1}^{N} \Phi_i$ can appear in SUSY-Lagrangian (No $\theta$ can appear in $\lag$, not "physical")!
\begin{subequations}
   \label{3.117}
\begin{align}
   \eqref{3.114a} &\Rightarrow \Phi_1 \Phi_2 = \phi_1 \phi_2 + \phi_1 \sqrt{2} \theta \xi_2 + \phi_2 \sqrt{2} \theta \xi_1 + \phi_1 \theta\theta F_2 + \phi_2 \theta\theta F_1 + 2 \theta^A \xi_{1A} \theta^B \xi_{2B} \label{3.117a}
   \shortintertext{with $\theta^A \xi_{1A}\theta^B \xi_{2B} = - \theta^A\theta^B \xi_{1A} \xi_{2B} \stackrel{(I7c)}{=} \frac{1}{2} \epsilon^{AB} \theta\theta \xi_{1A}\xi_{2B} = -\frac{1}{2}\xi_1 \xi_2 \theta \theta$} 
   \int \dd[2]{\theta} \Phi_1\Phi_2 &\stackrel{(I24a)}{=} \phi_1 F_2 + \phi_2 F_1 - \xi_1\xi_2 \label{3.117b}
\end{align}
\end{subequations}
Note the last term looks like fermion mass term!
\begin{equation}
   \int \dd[3]{\theta} \Phi_1 \Phi_2 \Phi_3 = F_1 \phi_2 \phi_3 + F_2 \phi_1 \phi_3 + F_3 \phi_1 \phi_2 - \xi_1 \xi_2 \phi_3 - \xi_1 \xi_3 \phi_2 - \xi_2 \xi_3 \phi_1 \label{3.118}
\end{equation}
The last three terms are Yukawas i.a..

Consider $\Phi_1^\dagger \Phi_2$, it is not chiral. Thus compute in $x$-space, \eqref{3.115}
\begin{align}
   \Phi_i^\dagger \Phi_k &= \phi_i^* \phi_k + \sqrt{2} \theta \xi_k \phi_i^* + \sqrt{2} \bar\theta \bar\xi_i \phi_k + \theta\theta\phi_i^* F_k + \bar\theta\bar\theta \phi_k F_i^* + 2 \bar\theta \bar\xi_i \theta \xi_k + \sqrt{2} \theta \theta \bar\theta_{\dot{A}} \left( i \bar\sigma^{\mu \dot{A}B} \xi_{k B} [\partial_\mu]\phi_i^* + \bar\xi_i^{\dot{A}} F_k \right) \notag \\ 
   &\quad + \sqrt{2} \bar\theta \bar\theta \theta^A \left( i \sigma^\mu_{A\dot{B}} \bar\xi_i^{\dot{B}} [\partial_\mu] \phi_k + \xi_{kA}F_i^* \right) - 2i \theta \sigma^\mu \bar\theta \phi_i^* [\partial_\mu] \phi_k \notag \\ 
   &\quad + \theta\theta\bar\theta\bar\theta \left( F_i^* F_k + \frac{1}{2} (\partial_\mu \phi_i^*)(\partial^\mu \phi_k) + \frac{1}{2} (\partial_\mu \phi_k^*)(\partial^\mu\phi_i) + i \xi_k \sigma^\mu [\partial_\mu] \bar\xi_i \right) \label{3.119}
\end{align}
with
\begin{equation}
   X[\partial_\mu] Y = \frac{1}{2} [X\partial_\mu Y - (\partial_\mu X) Y] \label{3.120}
\end{equation}
$\theta\theta\bar\theta\bar\theta$ component transforms into itself and total derivative. It can contribute to SUSY-invariant Lagrangian! For $i=k$, it looks like kinetic energy terms for scalar $\phi_i$ and fermion $\xi_i$!

\paragraph{WZ supergauge}
Another way to construct an irreducible representation of SUSY algebra demand superfield to be real
\begin{equation}
   V = V^\dagger \label{3.120p}
\end{equation}

Defines a \textit{vector superfield}. Component form of $V$, from general expression \eqref{3.94}
\begin{equation}
   V(x, \theta,\bar\theta) = c'(x) + \sqrt{2} \theta \xi'(x) + \sqrt{2} \bar\theta \bar{\xi'} (x) + \theta \theta M'(x) + \bar\theta\bar\theta M'^{*} (x) + \theta \sigma^\mu \bar\theta A'_\mu(x) + \theta\theta \bar\theta \bar{\lambda '} (x) + \bar\theta \bar\theta \theta \lambda'(x) + \frac{1}{2} \theta\theta\bar\theta \bar\theta D'(x) \label{3.121}
\end{equation}
with $c'(x), D'(x), A'_\mu \in \R$ and $M'(x) \in \Co$.
If $\Phi(x)$ is left-chiral, then $\Phi(x) + \Phi^\dagger (x) $ is vector superfield!

Let  
\begin{equation}
 i \Lambda (y,\theta) = \phi(y) + \sqrt{2} \theta \chi(y) + \theta\theta F(y)  
\end{equation}
then with expansion of $y$ into $x$
\begin{align}
   i \Lambda(x) - i \Lambda^\dagger (x) &= 2 \Re \phi(x) + \sqrt{2} \theta \chi + \sqrt{2} \bar\theta \bar\chi + \theta \theta F + \bar\theta \bar\theta F^* - 2 \theta \sigma^\mu \bar\theta \partial_\mu \Im \phi \notag \\
                                        &\quad - \frac{i}{\sqrt{2}} \theta\theta \bar\theta \bar\sigma^\mu \partial_\mu \chi - \frac{i}{\sqrt{2}} \bar\theta \bar\theta \theta \partial_\mu \bar\chi - \frac{1}{2} \theta \theta \bar\theta \bar\theta \partial^\mu \partial_\mu \Re\phi \label{3.122}
\end{align}

We want $A_\mu(x)$ to describe a physical gauge boson, $[A_\mu] = [E]^1$. Thus $V$ is dimensionless! Thus $[c] = [E]^0, [\xi] = [E]^{1/2}$ ? $[M] = [E]^1, [\lambda] = [E]^{3/2}$ is correct, $[D] = [E]^2$?

In non-SUSY QFT, abelian gauge transformation define throught a single real scalar field. In SUSY, it must be part of a complex scalar, which resides in a (left-)chiral superfield!

Comparing \eqref{3.122} with \eqref{3.121}: super gauge transformation given by
\begin{equation}
   V(x) \rightarrow V(x) + i (\Lambda(x) - \Lambda^\dagger(x)) \label{3.124}
\end{equation}
By choice of $\Re \phi, \chi, F$, we can gauge so that
\begin{equation}
   c(x) = \xi(x) = M(x) = 0 \label{3.125}
\end{equation}
The \textit{Wess-Zumino (WZ) supergauge} (see \eqref{3.127})! Note that we did not specify $\Im \phi$! 

\eqref{3.124} corresponds to 
\begin{equation}
   A_\mu (x) \rightarrow A_\mu (x) - 2 \partial_\mu \Im \phi (x) \label{3.126}
\end{equation}
a normal abelian gauge transformation! WZ supergauge can be combined with any "ordinary" gauge! 

\paragraph{Vector superfield in WZ supergauge}
\begin{equation}
   V_{WZ}(x) = \theta \sigma_\mu \bar\theta A^\mu (x) + \theta \theta \bar\theta \bar\lambda(x) + \bar\theta \bar\theta \theta \lambda(x) + \frac{1}{2} \theta\theta \bar\theta \bar\theta D(x) \label{3.127}
\end{equation}
Thus
\begin{subequations}
   \label{3.128}
\begin{align}
   V^2_{WZ} &= \frac{1}{2} \theta \theta \bar\theta \bar\theta A_\mu (x) A^\mu (x) \label{3.128a} \\
   V^n_{WZ} &= 0 \quad \forall n \leq 3 \label{3.128b}
\end{align}
\end{subequations}

Remarks
\begin{itemize}
   \item SUSY transformations \eqref{3.100} close on general vector superfield, \eqref{3.121}, with identification $\chi = \xi, N = M^*, \zeta = \lambda, f=c$ for example. \eqref{3.100a} $\Rightarrow \delta c \in \R$, etc..
   \item SUSY algebra does not close on off-shell vector superfield in WZ gauge. E.g.: \eqref{3.100d} generates $\delta M = \bar\epsilon \bar\lambda \neq = 0$ even if $M(x) = 0$ initially.

      Hence, specifying WZ supergauge means that \textit{manifest} SUSY invariance is lost. But we can restore WZ form after SUSY transformation by an \textit{additional} supergauge transformation. Similar to usual gauge theory, where manifest gauge invariance is lost once a gauge is specified.
   \item For a $\Uni(1)$ gauge symmetry, as discussed so far
      \begin{equation*}
         V_D = \int \underbrace{\dd[2]{\theta} \dd[2]{\bar\theta}}_{\dd[4]{\theta}} V_{WZ} (x, \theta, \bar\theta) = \int \dd[4]{\theta} V(x, \theta, \bar\theta) + \text{tot. dev.}
      \end{equation*}
      is both SUSY and gauge invariant (up to total derivatives). For left-chiral $\Phi_i, \Phi_k$, we will need
      \begin{subequations}
         \label{3.129}
      \begin{align}
         \Phi_i^\dagger V_{WZ} \Phi_k &= \theta \sigma^\mu \bar\theta A_\mu \phi_i^* \phi_k + \frac{1}{\sqrt{2}} \theta \theta \left( \bar\theta \bar\sigma^\mu \xi_k A_\mu \phi_i^* + \sqrt{2} \bar\theta \bar\lambda \phi_i^* \phi_k \right)  + \frac{1}{\sqrt{2}} \bar\theta \bar\theta \left( - \theta \sigma^\mu \bar\xi_i A_\mu \phi_k + \sqrt{2} \theta\lambda \phi_i^* \phi_k \right) \notag \\
                                      &\quad + \frac{1}{2} \theta \theta \bar\theta \bar\theta \left( D \phi_i^* \phi_k - 2i A_\mu \phi_i^* [\partial_\mu] \phi_k - \bar\xi_i \sigma^\mu \xi_k A_\mu - \sqrt{2} \phi_k \bar\lambda \bar\xi_i - \sqrt{2} \phi_i^* \lambda \xi_k \right) \label{3.129a} \\
         \Phi_i^+ V_{WZ} V'_{WZ} \Phi_k &= \frac{1}{2} \theta \theta \bar\theta \bar\theta A_\mu A'^{\mu} \phi_i^* \phi_k \label{3.129b}
      \end{align}
      \end{subequations}
      with fermion ($\xi$) , sfermion ($\phi$) and gaugino ($\lambda$).
\end{itemize}

We have found terms look like gauge interactions of matter fermions and scalars. Also need kinetic terms of gauge bosons! To that end, construct (abelian) field strength superfields
\begin{subequations}
\begin{align}
   W_A &= -\frac{1}{4} \bar\D \bar\D \D_A V \quad (\text{left-chiral}) \label{3.130a} \\
   \bar W_{\dot{A}} &= - \frac{1}{4} \D \D \D_{\dot{A}} V \quad (\text{right-chiral}) \label{3.130b}
\end{align}
\end{subequations}

Note 
\begin{itemize}
   \item $W_a$, $\bar W_{\dot{A}}$ are anti-commuting (fermionic) superfields, while vector superfield $V$ and chiral superfield $\Phi$ are bosonic (commuting).
   \item $[V] = [E]^0, [\D] = [E]^{1/2} \Rightarrow [W_A] = [\bar W_{\dot{A}}] = [E]^{3/2}$
   \item $W_A$, $\bar W_{\dot{A}}$ are invariant under supergauge transformation \eqref{3.124}
\begin{equation*}
   W_A \rightarrow W_A - \frac{i}{4} \bar\D \bar\D \D_A (\Lambda - \Lambda^\dagger)
\end{equation*}
Remember $\Lambda^\dagger$ is right-handed. Now just calculation the change
\begin{equation*}
   \delta W_A = \bar D_{\dot{B}} \bar\D^{\dot{B}} \D_A \Lambda = \bar\D_{\dot{B}} \acomm{\bar\D^{\dot{B}}}{\D_A} \Lambda = \acomm{\bar\D^{\dot{B}}}{\D_A} \bar\D_{\dot{B}} \Lambda = 0
\end{equation*}
$\bar\D$ commutes with the anti-commutator since the anti-commutator $\sim \sigma^\mu \partial_\mu $ according to \eqref{3.106b}. At last step, we used the fact that $\Lambda$ is left-chiral.

\item Explicit calculation of $W$ ($\bar W$) is more convenient in left-(right-) chiral parametrization of superspace
   \begin{align*}
      y^\mu &= x^\mu - i \theta \sigma^\mu \bar\theta \\
      \bar y^\mu &= x^\mu + i \theta \sigma^\mu \bar\theta \\
      V_{WZ}(y, \theta, \bar\theta) &= \theta \sigma^\mu \bar\theta A_\mu(y) + \theta\theta \bar\theta \bar\lambda(y) + \bar\theta \bar\theta \theta \lambda(y) + \frac{1}{2} \theta \theta \bar\theta \bar\theta \left[ D(y) + \partial_\mu^{(y)} A^\mu (y) \right] \\
      V_{WZ}(\bar y, \theta, \bar\theta) &= \theta \sigma^\mu \bar\theta A_\mu(\bar y) + \theta\theta \bar\theta \bar\lambda( \bar y) + \bar\theta \bar\theta \theta \lambda(\bar y) + \frac{1}{2} \theta \theta \bar\theta \bar\theta \left[ D(\bar y) - \partial_\mu^{(\bar y)} A^\mu (\bar y) \right]
   \end{align*}
   We have used $(I7k)$ in expansion of $A_\mu (x)$. Since $W_A$ is supergauge invariant, it is sufficient to use $V$ in WZ supergauge, without loss of generality.
\end{itemize}

To calculate $W_A$, use left-chiral form of $\D, \bar \D$ \eqref{3.112}:
\begin{subequations}
   \label{3.132}
\begin{align}
   \D_A^{(y)} V_{WZ} (y, \theta, \bar\theta) &= \sigma^\mu_{A\dot{B}} \bar\theta^{\dot{B}} A_\mu (y) + 2 \theta_A \bar\theta \bar\lambda(y) + \bar\theta \bar\theta \lambda_A(y) + \bar\theta \bar\theta \left[ \delta_A^B D(y) - \sigma^{\mu\nu B}_A F_{\mu\nu}(y) \right] \theta_B \notag \\ 
                                             &\quad + i \theta \theta \bar\theta \bar\theta \left(\sigma^\mu \partial_\mu^{(y)} \bar\lambda(y)\right)_A \label{3.132a} \\
   W_A (y) &= \lambda_A (y) + D(y) \theta_A - (\sigma^{\mu\nu} \theta)_A F_{\mu\nu}(y) + i\theta \theta \sigma^\mu_{A\dot{B}} \partial_\mu^{(y)} \bar\lambda^{\dot{B}} (y) \label{3.132b} \\
   \bar W_{\dot{A}} (\bar y) &= \bar\lambda_{\dot{A}} (\bar y) + D(\bar y) \bar \theta _{\dot{A}} - \epsilon_{\dot{A}\dot{B}} (\bar\sigma^{\mu\nu}\bar\theta)^{\dot{B}} F_{\mu\nu}(\bar y) - i \bar\theta \bar\theta \partial_\mu^{(\bar y)} \lambda^B (\bar y) \sigma^\mu_{B \dot{A}} \label{3.132c}
\end{align}
\end{subequations}
From these
\begin{subequations}
   \label{3.133}
\begin{align}
   W^A W_A &= \lambda(y) \lambda(y) + 2 \theta \left[ D(y) \lambda(y) + \sigma^{\mu\nu} \lambda(y) F_{\mu\nu}(y) \right] \notag \\ 
           & \quad + \theta \theta \left[ D^2(y) + 2i \lambda(y) \sigma^\mu \partial_\mu^{(y)} \bar\lambda(y) - \frac{1}{2} F_{\mu\nu} F^{\mu\nu} - \frac{i}{2} \tilde F_{\mu\nu} F^{\mu\nu} \right] \label{3.133a} \\
   \bar W_{\dot{A}} \bar W^{\dot{A}} &=  \lambda(y) \lambda(y) + 2 \theta \left[ D(y) \lambda(y) + \sigma^{\mu\nu} \lambda(y) F_{\mu\nu}(y) \right] \notag \\
                                     &\quad + \bar\theta \bar\theta \left[ D^2(\bar y) + 2i \lambda(\bar y) \sigma^\mu \partial_\mu^{(\bar y)} \bar\lambda(\bar y) - \frac{1}{2} F_{\mu\nu} F^{\mu\nu} - \frac{i}{2} \tilde F_{\mu\nu} F^{\mu\nu} \right] \label{3.133b}
\end{align}
\end{subequations}
with $F_{\mu\nu}$ field strength tensor and $\tilde F_{\mu\nu} = \frac{1}{2} \epsilon_{\mu\nu\alpha \beta} F^{\alpha \beta}$ dual field strength tensor. In $\theta\theta$ or $\bar\theta \bar\theta$ component, one can replace $y,\bar y \rightarrow x$ without additional terms.
\begin{equation}
   \frac{1}{4} \left[ \int \dd[2]{\theta W^A W_A} + \int \dd[2]{\bar\theta} \bar W_{\dot{A}} \bar W^{\dot{A}} \right] = \frac{1}{2} D^2 (x) - \frac{1}{4} F_{\mu\nu} (x) F^{\mu\nu} (x) + i\lambda(x) \sigma^\mu [\partial_\mu] \bar\lambda(x) \label{3.134}
\end{equation}
It contains properly normalized kinetic energy terms for the vector and gaugino component fields. Note that no derivative is acting on "auxiliary" $D$ component field.

\paragraph{$R$ parity and matter parity}
\begin{subequations}
   \label{3.135}
\begin{align}
   \eqref{3.78f}: R Q_A &= Q_A R - Q_A \notag \\
   \Rightarrow Q_A &\stackrel{R}{\rightarrow} e^{i\phi R} Q_A e^{-i\phi R} = e^{-i\phi} Q_A \label{3.135a} \\
   \eqref{3.78g} \Rightarrow \bar Q_{\dot{A}} &\rightarrow e^{i\phi} \bar Q_{\dot{A}} \label{3.135b} \\
   \eqref{3.96} \Rightarrow \theta_A &\rightarrow e^{i\phi} \theta_A, \bar\theta_{\dot{A}} \rightarrow e^{-i\phi} \bar\theta_{\dot{A}}
\end{align}
\end{subequations}
Define $R$ charge
\begin{equation}
   R(\theta) = R(\bar Q) = +1, \quad R(\bar \theta) = R(Q) = -1 \label{3.136}
\end{equation}

For left-chiral superfield
\begin{subequations}
   \label{3.137}
\begin{align}
   \Phi(x, \theta,\bar\theta) &\rightarrow \Phi'(x, e^{i\phi}\theta, e^{-i\phi}\bar\theta ) \stackrel{!}{=} e^{i\phi R} \Phi(x, \theta, \bar\theta) \label{3.137a} \\
   \Phi^\dagger (x, \theta, \bar\theta) &\rightarrow \Phi'^\dagger (x,\theta,\bar\theta) = e^{-i\phi R} \Phi^\dagger (x, \theta, \bar\theta) \label{3.137b}
\end{align}
\end{subequations}
Comparing to \eqref{3.114}, this implies
\begin{subequations}
   \label{3.137}
\begin{align}
   R(\phi) &= R(\Phi) \label{3.138a} \\
   R(\xi) &= -R(\bar\xi) = R(\Phi) - 1 \label{3.138b} \\
   R(F) &= R(\Phi) - 2 \label{3.138c}
\end{align}
\end{subequations}
$R$ charge of products of chiral superfields adds!

\begin{subequations}
   \label{3.139}
\begin{align}
   V = V^\dagger \Rightarrow R(V) &= 0 \label{3.139a} \\
   R(A_\mu) &= 0 \label{3.139b}\\
   R(\lambda) &= - R(\bar\lambda) = +1 \label{3.139c}\\
   R(D) &= 0 \label{3.139d}
\end{align}
\end{subequations}

Might this implies $R$-invariance be a symmetry of Nature?
\begin{itemize}
   \item Matter kinetic term $\Phi_i^\dagger \Phi_i$ invariant (from \eqref{3.137}).
   \item Gauge interaction terms $\Phi^\dagger_i V \Phi_i$, $\Phi_i^\dagger V V \Phi_i$ invariant (from \eqref{3.137}, \eqref{3.139}).
   \item Yukawa interactions may or may not be $R$-invariant: depends on $R$-charges of involved superfields.
   \item Gaugino mass terms are not $R$ invariant, $m_\lambda \lambda \lambda$ has $R$-charge $+2$!
\end{itemize}

However, it is invariant under discrete ($Z_2$) subgroup of $U(1)_R$ transformation: \eqref{3.135}-\eqref{3.137} with the phase $\phi \in \{0, \pi \}$.

Define matter parity $(-1)^{R(\Phi)}$. R parity is the corresponding value for component fields. Thus
\begin{itemize}
   \item
   \begin{itemize}
      \item gauge bosons are $R_p$ even \eqref{3.139b} (spin-1)
      \item gauginos are $R_p$ odd (spin-$1/2$)
   \end{itemize}
   \item chiral Higgs superfields are conventionally assigned $R=0$
      \begin{itemize}
         \item  Higgs bosons are $R_p$ even (spin-$0$)
         \item Higgsinos are $R_p$ odd (spin-$1/2$)
      \end{itemize}
   \item chiral matter superfields: assign unit charge, $R=1$
      \begin{itemize}
         \item sfermions are $R_p$ odd (spin-$0$): squarks, sleptons
         \item fermions are $R_p$ even (spin-$1/2$)
      \end{itemize}
\end{itemize}
Hence by design: SM particles are $R_p$ even, their superpartners are odd. Within the SM, we can write
\begin{equation}
   R_p = (-1)^{3(B-L)+2S} \label{3.140}
\end{equation}

$R$ parity may or may not be conserved in nature. 
\begin{itemize}
   \item Lightest superparticle (LSP) (lightest $R$-odd particles) is stable! By energy conservation, it cannot decay into an $R$-odd final state, which would need to contain odd number of superfields.
   \item Starting from SM particles, i.e.~$R$-even state, superparticles can only produced in pairs, so that final state is also $R$-even. Cannot produce a single superparticle starting from SM particles.
   \item Any stable particle has to be neutral (from searches for exotic isotopes). If $R_p$ is conserved, LSP must be neutral. Thus missing energy signature for collider searches, if heavier superparticles decay inside the detector.
   \item Stable LSP might be candidate for cosmological Dark Matter.
\end{itemize}
It frequently does not work in extensions of the SM, e.g. in SUSY $\SU(5)$.


\section{Interacting superfields: Constructing SUSY Lagrangians}
\subsection{System of interacting chiral superfields}
Consider system of left-chiral superfields $\{\Phi_i\}$. We had seen that SUSY-invariant contributions to $\lag$ can be
\begin{itemize}
   \item $\prod_{i} \Phi_i |_{\theta\theta}$ (only take $\theta\theta$ term) see \eqref{3.116c}, \eqref{3.117}, \eqref{3.118}
   \item $\Phi_i^\dagger \Phi_k|_{\theta\theta\bar\theta\bar\theta}$ see \eqref{3.100i}, \eqref{3.119}
\end{itemize}
Most general ansatz for power-counting renormalizable Lagrangian
\begin{equation*}
   \lag = \int \dd[4]{\theta} \sum_{i,k} \M_{ik} \Phi_i^\dagger \Phi_k + \left[ \int \dd[2]{\theta} W(\Phi_i) + h.c. \right]
\end{equation*}
$\lag$ must be hermitian and $\M$ must be hermitian, The basis can be chosen as $\M = \id$
\begin{equation}
   \lag = \int \dd[4]{\theta} \sum_i \Phi^\dagger_i \Phi_i + \left[ \int \dd[2]{\theta} W(\Phi_i) + h.c. \right] \label{3.141}
\end{equation}
Recall that $[\int \dd[2]{\theta}] = [E]^1$, $[\int \dd[4]{\theta} ] = [E]^2$, $[\Phi_i] = [E]^1$.
No higher power of $\Phi_i^\dagger \Phi_i $ are allowed in renormalizable theory. Superpotential $W(\Phi_i)$ must be polynomial of degree $\leq 3$.
\begin{equation}
   W(\Phi_i) = h_i \Phi_i + \frac{1}{2} m_{ij} \Phi_i \Phi_j + \frac{1}{3!} f_{ijk} \Phi_i \Phi_j \Phi_k \label{3.142}
\end{equation}
with $[h_i] = [E]^2$, $[m_{ij}] = [E]^{1}$, $[f_{ijk}]=[E]^0$.

Define
\begin{subequations}
\label{3.143}
\begin{align}
   W_i (\phi) &=  \eval{\pdv{W}{\Phi_i}}_{\theta=\bar\theta=0}  \label{3.143a} \\
   W_{ij} (\phi) &= \eval{\pdv{W}{\Phi_i}{\Phi_j}}_{\theta=\bar\theta=0}  \label{3.143a} \\
   W_{ijk} (\phi) &= \eval{\frac{\partial^3 W}{\partial \Phi_i \Phi_j \Phi_k}}_{\theta=\bar\theta=0}  \label{3.143a} \\
   W^i (\phi^*) &= \eval{\pdv{W^\dagger}{\Phi_i^\dagger}}_{\theta=\bar\theta=0} \label{3.143d} \\
   \text{etc.} \notag
\end{align}
\end{subequations}

Write \eqref{3.141} in component form
\begin{alignat*}{2}
   \lag &= i \xi_i [\partial_\mu] \sigma^\mu \bar\xi_i + \partial_\mu \phi_i^* \partial^\mu \phi_i + F_i^* F_i && \leftarrow \eqref{3.119}\\
        &\quad +\Big[ h_i F_i &&\leftarrow \eqref{3.114a} \\
        &\quad +\frac{1}{2} m_{ij} (\phi_i F_j + \phi_j F_i - \xi_i \xi_j) &&\leftarrow \eqref{3.117b}\\
        &\quad + \frac{1}{6} f_{ijk} (F_i \phi_j \phi_k + F_j \phi_i \phi_k + F_k \phi_i \phi_j - \xi_i \xi_j \phi_k - \xi_i\xi_k \phi_j - \xi_j \xi_k \phi_i) &&\leftarrow \eqref{3.118} \\
        & \quad + h.c. \Big]  &&
\end{alignat*}
Note that $m_{ij}$ and $f_{ijk}$ must be totally symmetric (summation!).
\begin{equation}
   \lag = i \xi_i [\partial_\mu] \sigma^\mu \bar\xi_i + \partial_\mu \phi_i^* \partial^\mu \phi_i + F_i^* F_i + \left[ h_i F_i + m_{ij} \left( \phi_i F_j - \frac{1}{2} \xi_i \xi_j \right) + \frac{1}{2} f_{ijk} \left( F_i \phi_j \phi_k - \phi_i \xi_j \xi_k \right) + h.c. \right] \label{3.144}
\end{equation}
Note that it does not contain derivatives acting of $F_i$, since $\partial_\mu F_i$ can be removed by using the equations of "motion" (better: equations of constraint, since the $F_i$ do not propagate!)
\begin{align}
   \pdv{\lag}{F_i} &= h_i + m_{ij} \phi_j + \frac{1}{2} f_{ijk} \phi_j \phi_k + F_i^* =: W_i (\phi) + F_i^* = 0 \notag \\
   \Rightarrow F_i^* &= - W_i,\quad F_i = - \bar W^i \label{3.145}
\end{align}
Insert into \eqref{3.144}
\begin{align}
   \lag &= i \xi_i [\partial_\mu] \bar\xi_i + \partial_\mu \phi_i^* \partial^\mu \phi_i + W_i \bar W^i - \bar W_i W^i - W_i \bar W^i - \frac{1}{2} \left[ \xi_i \xi_j (m_{ij} + f_{ijk}\phi_k ) +h.c. \right] \notag \\
   \Rightarrow \lag &= i\xi_i [\partial_\mu ] \sigma^\mu \bar\xi_i + \partial_\mu \phi_i^* \partial^\mu \phi_i - \sum_i |W_i|^2 - \left(\sum_{i,j} \frac{1}{2} \xi_i \xi_j W_{ij} + h.c. \right) \label{3.146}
\end{align}

Scalar potential is the "F-term" on-shell
\begin{subequations}
   \label{3.147}
\begin{align}
   V(\phi_i) &= \sum_i  \eval{\left|\pdv{W}{\Phi_i}\right|^2}_{\theta=\bar\theta=0}   = \sum_i |F_i|^2 \label{3.147a}
   \shortintertext{Fermion masses and Yukawa couplings come from $W_{ij}$. Thus fermion mass matrix}
   m_{ij}^{(f)} &= \expval{W_{ij}} = W_{ij} (\expval{\phi_k}) \label{3.147b} 
\end{align}
\end{subequations}

Simplest example is the Wess-Zumino model. It has single chiral superfield
\begin{align}
   \Phi &= (\phi, \xi), \quad W=h\Phi + \frac{1}{2} m \Phi^2 + \frac{1}{6} f \Phi^3 \notag \\ 
   \Rightarrow \lag &= i \xi [\partial_\mu] \partial^\mu \bar\xi + \partial_\mu \phi^*\partial^\mu \phi - \frac{1}{2} m (\xi \xi + \bar\xi\bar\xi) - \frac{f}{2} (\xi \xi \phi + \bar\xi\bar\xi \phi^*) - \left| h + m \phi + \frac{f}{2} \phi^2 \right|^2 \label{3.148}
\end{align}

It has only one $2$-component spinor, thus can form a Majorana $4$-spinor.
\begin{align*}
   &\eqref{3.67} \Rightarrow \psi_M = \begin{pmatrix} \xi_\vee \\ \bar\xi^\wedge \end{pmatrix}  \Rightarrow \bar\psi_M = (\xi^\wedge, \bar\xi_\vee) \\
   &\Rightarrow \frac{1}{2} \xi \sigma^\mu \partial_\mu \bar\xi - \frac{1}{2} (\partial_\mu \xi) \sigma^\mu \bar\xi \stackrel{(I.6a)}{=} \frac{1}{2} \left[ \xi \sigma_\mu \partial_\mu \bar\xi + \bar\xi \bar\sigma^\mu \partial_\mu \xi \right] \\
   &\quad \stackrel{\eqref{0.5}, \eqref{3.37}}{=} \frac{1}{2} \bar\psi_M \gamma^\mu \partial_\mu \psi_M
\end{align*}

\paragraph{Minimal Supersymmetric Standard Model (MSSM)}
Recall that we are only using left-chiral superfields, thus we also need left-handed fermion representation \eqref{2.16}. Use capital letters for superfields
\begin{equation}
   Q = (\tilde q_L, q_L); \; \tilde U = (\tilde u^*_R, (u_R)^c); \; \tilde D = (\tilde d_R^*, (d_R)^c); \; L = (\tilde{l}_L, l_L); \; \tilde E = (\tilde e_R^*, (e_R)^c) \label{3.150}
\end{equation}
Note that 
\begin{itemize}
   \item They have the following charges
   \begin{itemize}
   \item $\tilde E$ has charge $+1$, $\tilde e_R$ has charge $-1$ 
   \item $\tilde U$ has charge $-\frac{2}{3}$, $\tilde u_R$ has charge $+\frac{2}{3}$
   \item $\tilde D$ has charge $+\frac{1}{3}$, $\tilde d_R$ has charge $-\frac{1}{3}$
   \end{itemize}
\item subscripts $L$, $R$ on scalars do not (of course) denote chirality.  It simply distinguishes $\SU(2)$ doublets ($L$) from singlets ($R$).
\item We have suppressed generation indices in \eqref{3.150}: all matter superfields come in (at least) $3$ generations.
\end{itemize}

Recall that in \eqref{3.144} $\lag \supset -\frac{1}{2} f_{ijk} \phi_i \xi_k \xi_k$ has the same gauge structure as the superpotential. \textit{Thus the superpotential must be gauge invariant!}

In MSSM, Yukawa interactions must come from superpotential. Recall in SM, we need both Higgs doublet $\phi_h$ ($Y=+\frac{1}{2}$) and its conjugate $\tilde \phi_h = i \sigma^2 \phi^\dagger$ ($Y=-\frac{1}{2}$). In SUSY, if $\phi_h$ is in a left-chiral superfield, $\phi_h^\dagger$ is in right-chiral superfield. This is not allowed in superpotential! We need two Higgs doublet superfields in the MSSM, with opposite hypercharges.
\begin{align}
   \begin{split}
      H_1 &= (h_1, \tilde{h}_1), Y_{H_1} = -\frac{1}{2}   \\
      H_2 &= (h_2, \tilde{h}_2), Y_{H_1} = +\frac{1}{2}   
   \end{split}
   \label{3.151}
\end{align}
Thus
\begin{equation}
   W_\text{MSSM} = f_{ik}^{(e)} H_1 \cdot L_i \bar E_k + f_{ik}^{(d)} H_1 \cdot Q_i \bar D_k + f_{ik}^{(u)} Q_i \cdot H_2 \bar U_k + \mu H_1 \cdot H_2 \label{3.152}
\end{equation}

Remarks
\begin{itemize}
   \item Yukawa couplings $f^{(e)}$, $f^{(d)}$, and $f^{(u)}$ have same structure as in SM, but different numerical values.

      We have 
      \begin{equation*}
         M_W = \frac{g}{\sqrt{2}} \sqrt{ \expval{h_1^0}^2 + \expval{h_2^0}^2  } = \sqrt{v_1^2 + v_2^2} = v \approxeq \SI{175}{\giga \eV}
      \end{equation*}
      with usual notation
      \begin{equation}
         v_1 =  v\cos \beta,\quad v_2 = v \sin \beta \label{3.153}
      \end{equation}

      Masses of SM fermions are fixed 
      \begin{equation}
         \eval{f^{(e)}}_\text{MSSM} = \eval{f^{(e)}}_\text{SM} \cdot \frac{1}{\cos \beta}, \quad 
         \eval{f^{(d)}}_\text{MSSM} = \eval{f^{(d)}}_\text{SM} \cdot \frac{1}{\cos \beta}, \quad
         \eval{f^{(u)}}_\text{MSSM} = \eval{f^{(u)}}_\text{SM} \cdot \frac{1}{\sin \beta} \label{3.154}
      \end{equation}
      Yukawa couplings are larger! $\tan \beta \gg 1 \Rightarrow \frac{1}{\cos \beta} \gg 1$ and $f^{(b)} \approx f^{(t)}$ in MSSM.
   \item Have included one mass term $\mu$ in \eqref{3.152}. It is allowed by all symmetries and needed for phenomenology. Only mass paramter in MSSM that conserves SUSY!
   \item $W_\text{MSSM}$ is linear in Higgs superfields (except for the mass term). Thus $\sum_i \left| \pdv{W_\text{MSSM}}{\Phi_i}| \right|^2 $ does not contain quartic Higgs self interactions. We cannot yet discuss Higgs mechanism (EW symmetry breaking).
   \item From $f^{(t)} Q_3 \cdot H_2 \bar T$ we get quartic couplings
      \begin{align*}
         \lag &\supset - |f^{(t)}|^2 \cdot \left[ \left| \tilde q_{L3} \cdot h_2 \right|^2 + \left|\tilde t_R \right|^2 \left( |h_2^\dagger|^2 + |h_2^0|^2 + |\tilde t_L|^2 + |\tilde b_L|^2 \right) \right] \\
              &= - |f^{(t)}|^2 \cdot \left[ \left| \textcolor{red}{\tilde t_L h_2^0} - \tilde b_L h_2^\dagger \right|^2 + \left|\tilde t_R \right|^2 \left( |h_2^\dagger|^2 + \textcolor{red}{|h_2^0|^2} + |\tilde t_L|^2 + |\tilde b_L|^2 \right) \right] \\
      \end{align*}
      Terms in \textcolor{red}{red} are needed to cancel quadratic divergences in $m^2_{h_t}$ from top loops, see \eqref{3.13}.
   \item $H_1$ and $L_i$ have same gauge quantum numbers: why can't we use one of $L_i$ to replace $H_1$? Reason is that adding only $\tilde{h}_2$ to SM fermions leads to gauge anomalies, e.g.~
      \begin{equation*}
         \feynmandiagram[horizontal=i to v1, medium, tree layout]{
            i[particle={\(B, W_3\)}] --[photon] v1 --[edge label={\(\tilde h_2\)}] v2 --[photon] f1[particle={\(B, W_3\)}],
            v1 -- v3 --[photon] f2[particle={\(B, W_3\)}],
            v3 -- v2,
         };
      \end{equation*}
      doesn't vanish (for $B^3$, $B W_3^2$). (Gauginos are anomaly-free by themselves!) Thus we need $\tilde h_1$ with opposite hypercharge!
   \item $W_\text{MSSM}$ in \eqref{3.152} respects $B$ and $L$, just like SM does. In SM, it is automatic (more like accidental) consequence of gauge group and matter content. Thus cannot write renormalizable term that breaks $B$ or $L$. In MSSM, we can write such terms
      \begin{equation}
         W_{\slashed{R}_p} = \frac{1}{2} \lambda_{ijk} L_i L_j \bar E_k + \lambda'_{ijk} L_i \cdot Q_j \bar D_k + \frac{1}{2} \lambda''_{ijk} \bar U_{i} \bar D_{j} \bar D_{k} - \epsilon_i L_i \cdot H_2 \label{3.155}
      \end{equation}
      with 
      \begin{equation}
       \lambda_{ijk} = -\lambda_{jik}, \quad
       \lambda''_{ijk} = -\lambda''_{ikj} \label{3.156}
      \end{equation}
      $\lambda, \lambda', \epsilon$ break $L$ and $\lambda''$ breaks $B$. If \textit{both} $B$ and $L$ are broken, proton would decay very rapidly and it would be disaster. We can only one of $B$ and $L$ to be broken: $\lambda \cdot \lambda'', \lambda'\cdot \lambda'', \epsilon\cdot\lambda''=0$. All therms in \eqref{3.155} break $R_p$. None of the terms in \eqref{3.155} is needed for phenomenology. In this lecture, apply Occam's razor (just impose $R_p$ conservation)!
   \item $W_\text{MSSM}$ gives $m_{\nu_i} = 0$ exactly, just like \eqref{0.14} in SM. We can e.g.~implement see-saw, or write non-renormalizable terms as in \eqref{1.41}:
      \begin{equation}
         W_{\nu-\text{mass}} = -\frac{1}{2} L_i \cdot H_n \kappa_{ik} L_k \cdot H_n \label{3.157}
      \end{equation}
      It breaks $L$ but does not break $R_p$!
\end{itemize}
To complete construction of MSSM, we need supersymmetric treatment of (non-abelian) gauge interactions!

\paragraph{Abelian gauge interactions}
Define supergauge transformation of chiral superfields

\begin{subequations}
\label{3.158}
\begin{align}
   \Phi_k &\rightarrow e^{-2i g t_k \Lambda(t)} \Phi_k, \quad \bar\D^{\dot{A}}\Lambda = 0 \label{3.158a} \\
   \Phi_k^\dagger &\rightarrow \Phi^\dagger e^{2i g t_k \Lambda^\dagger(t)}, \quad \D_A\Lambda^\dagger = 0 \label{3.158b}
\end{align}
\end{subequations}
with $t_k \in \R$ the charge of superfield $\Phi_k$, $g \in \R$ the gauge couplings. Put factor of $2$ in exponent, since $2\partial_\mu \Im \phi$ appears in transformation of $A_\mu$, see \eqref{3.126}. $\Lambda$ is left-chiral superfield, but $\Lambda \neq \Lambda^\dagger$.

Term in \eqref{3.141}
\begin{equation*}
   \int \dd[4]{\theta} \sum_k \Phi^\dagger_k \Phi_k 
\end{equation*}
is \textit{not} gauge invariant! Remember kinetic energy term in non-SUSY theory is not gauge invariant either! (That's why we introduced covariant derivatives!) But 
\begin{equation*}
   \Phi^\dagger_k e^{2gt_k V} \Phi_k \stackrel{\eqref{3.158}, \eqref{3.124}}{\rightarrow} \Phi^\dagger_k e^{2igt_k \Lambda^\dagger} e^{2gt_k(V + i\Lambda - i \Lambda^+)} e^{-2igt_k \Lambda} \Phi_k =  \Phi^\dagger_k e^{2gt_k V} \Phi_k 
\end{equation*}
is invariant and real ($g, t_k \in \R, V=V^\dagger$).

With this information, we can write $\Uni(1)$ gauge and SUSY invariant Lagrangian
\begin{equation}
   \lag = \int \dd[4]{\theta} \left( \Phi_k^\dagger e^{2gt_k V} \Phi_k + 2\eta V \right) + \left[ \int \dd[2]{\theta} \left( \frac{1}{4} W^A W_A + W(\Phi_i) \right) +h.c. \right] \label{3.159} 
\end{equation}
Recall that $\int \dd[4]{\theta} V = D$ is both SUSY- and $\Uni(1)$ invariant. $[\eta] = [E]^2$ is the \textit{Fayet-Illionpoulos term}. Expand the first term in \eqref{3.159} 
\begin{equation*}
   \Phi_k^\dagger e^{2gt_k V} \Phi_k = \Phi_k^\dagger \Phi_k + \underbrace{2gt_k \Phi_k^\dagger V \Phi_k + 2 g^2t_k^2 \Phi^\dagger_k V^2 \Phi_k}_{\eqref{3.129}} + \order{g^3 V^3}
\end{equation*}
in which the higher order terms vanish in WZ supergauge.

Hence, in WZ supergauge, using \eqref{3.144} for non-gauge terms
\begin{align}
   \lag &= \underbrace{-\frac{1}{4} F_{\mu\nu} F^{\mu\nu} + i\lambda \sigma^\mu \partial_\mu \bar\lambda + \frac{1}{2} D^2 + \eta D}_{\eqref{3.134}} + i\xi_k \sigma^\mu (\partial^\mu - igt_k A^\mu) \bar\xi_k + \left| (\partial_\mu + igt_k A_\mu)\phi_k \right|^2 - F^*_k F_k \notag \\
   &\quad- \left( \frac{1}{2} \xi_i \xi_j W_{ij}(\phi) + h.c. \right) - \sqrt{2} g t_k \left( \bar\lambda \bar\xi_k \phi_k + h.c. \right) + gt_k |\phi_k|^2 D \label{3.160}
\end{align}
We have used
\begin{equation}
   i\lambda \sigma^\mu [\partial_\mu] \bar\lambda = i\lambda \sigma^\mu \partial_\mu \bar\lambda + \text{tot. derivative} \label{3.161}
\end{equation}


Note
\begin{itemize}
   \item Did not generate any new terms involving $F_k$. $F$-term contribution \eqref{3.147a} to scalar potential remains unchanged.
   \item \eqref{3.160} does not depend on derivatives of $D$. Equation of constraints
      \begin{align}
         &\pdv{\lag}{D} \stackrel{!}{=} 0 = D + \eta + g \sum_k t_k |\phi_k|^2 \notag \\ 
         &\Rightarrow D = -\eta -g \sum_k t_k |\phi_k|^2 \label{3.162}
      \end{align}
      Insert this into \eqref{3.160} generates a new contribution to the scalar potential: "$D$-term" contribution
      \begin{align}
         V_D &= -\frac{1}{2} D^2 - \eta D -g \sum_k t_k |\phi_k|^2 D \stackrel{\eqref{3.162}}{=} -\frac{1}{2} D^2 - \eta D + D(D + \eta) = + \frac{1}{2} D^2 \notag \\
         \stackrel{\eqref{3.162}}{\Rightarrow} V_D &= \frac{1}{2} \left( \eta + g \sum_k t_k |\phi_k|^2 \right)^2 \label{3.163}
      \end{align}
\end{itemize}

   Generalization to several $\Uni(1)$ factors, with coupling $g_a$, is straightforward. We need kinetic energy terms for $A_\mu^a$ and gauginos $\lambda^a$, more terms in gauge-covariant derivatives: 
   \begin{align}
      \begin{split}
         D_\mu &= \partial_\mu - i \sum_a g_a t^a_k A^\mu_a \\
         V_D &= \frac{1}{2} \sum_a D^2_a \\
         D_A &= - \eta_a - g_a \sum_k \phi_k^* t_k^a \phi_k
      \end{split}
      \label{3.164}
   \end{align}

\paragraph{Non-abelian gauge interactions}
\section{Models of SUSY breaking}
\section{The minimal supersymmetric Standard Model (MSSM)}
\section{Phenomenological supergravity}
\section{Gauge-mediated SUSY breaking}
\section{Extensions of the MSSM}
